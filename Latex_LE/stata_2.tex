%----------------------------------------------------------
% PACKAGES AND THEMES
%----------------------------------------------------------
\documentclass[aspectratio=169,xcolor=dvipsnames,handout]{beamer}

\usetheme{Darmstadt}
\usecolortheme{seahorse}

\usepackage[hangul]{kotex}
\usepackage{hyperref}
\usepackage{graphicx, xcolor, amsmath}
\usepackage{booktabs, array, adjustbox, makecell, multicol, multirow}

\setbeamercovered{transparent}

%----------------------------------------------------------
% TITLE PAGE
%----------------------------------------------------------
\title{코딩의 기초}
\author{오성재}
\institute[CNU]
{\relax
    전북대학교 Stata 특강\
}
\date{2024년 10월 24일}

%----------------------------------------------------------
\begin{document}
%----------------------------------------------------------

\frame{\titlepage}

\begin{frame}
\frametitle{학습목표}
    \begin{itemize}[<+->]
    \item 첫 시간은 Stata에서 사용하는 dta 파일의 자료를 살펴보는 법을 익혔다.
    \item 이번 시간의 목표는 dta 파일을 가공하는 법을 익히는 것이다.
    \end{itemize}
\end{frame}

\begin{frame}{목차}
    \small
    \tableofcontents[hideallsubsections]
\end{frame}

\section{편리한 환경 만들기}

\begin{frame}[fragile]
    \frametitle{폴더관리}
    \begin{itemize}[<+->]
        \item dropbox: 자료공유 및 접근 용이.
        \begin{verbatim}
            ~/dropbox/data/your_data/rawdata
        \end{verbatim}
        \item 폴더명은 영어로: 영어 이외의 문자를 폴더명으로 할 경우 인식문제 발생. 오랜 관습
        \item 스네이크 케이스 (snake case)사용 추천: 파일 및 폴더명의 공백은 `\_' 로 대체. 공백으로 둘 경우 불편함 발생
    \end{itemize}
\end{frame}

\begin{frame}[allowframebreaks]
    \frametitle{Github}
    \begin{enumerate}[<+->]
        \item GitHub이란 무엇인가?
        \begin{itemize}[<+->]
            \item Git: 코드 버전 관리를 위한 도구로, 코드를 저장하고 변경 이력을 추적할 수 있습니다.
            \item GitHub: Git을 기반으로 하는 클라우드 플랫폼으로, 팀원들과 협업하여 코드 프로젝트를 관리하고 공유할 수 있습니다.
            \item 주요 특징:
            \begin{itemize}[<+->]
                \item 저장소 (Repository): 프로젝트 파일과 변경 기록을 저장하는 공간.
                \item 버전 관리: 코드의 과거 상태를 보관하고, 이전 버전으로 쉽게 돌아갈 수 있음.
                \item 협업: 팀원들이 각자 작업한 코드를 쉽게 통합하고 관리할 수 있음.
            \end{itemize}
        \end{itemize}
        \framebreak%
        \item GitHub의 기본 개념
        \begin{itemize}[<+->]
            \item Commit: 코드를 변경한 후 저장하는 작업, 변경 이력이 기록됩니다.
            \item Branch: 새로운 기능을 개발할 때 기존 코드와 분리된 별도의 작업 공간을 만듭니다.
            \item Pull Request: 자신이 작업한 코드를 다른 팀원이 검토할 수 있도록 요청하는 과정.
            \item Fork \& Clone: 다른 사람의 프로젝트를 내 저장소로 복사하거나 로컬 컴퓨터에 가져와 수정할 수 있습니다.
        \end{itemize}
        \item GitHub의 이점:
        \begin{itemize}[<+->]
            \item 안전한 코드 보관
            \item 효율적인 협업
            \item 자동화된 배포 및 테스트
        \end{itemize}
    \end{enumerate}
\end{frame}

\begin{frame}[allowframebreaks]
    \frametitle{Text Editor}
    \begin{itemize}[<+->]
        \item Text Editor란 무엇인가?
        \begin{itemize}[<+->]
            \item 텍스트 파일을 작성하고 수정할 수 있는 프로그램입니다.
            \item 코드 작성, 노트 기록, 메모 등을 간편하게 할 수 있습니다.
        \end{itemize}
        \item 왜 Text Editor를 사용해야 하나요?
        \begin{itemize}[<+->]
            \item 빠르고 가볍습니다: 빠르게 실행되며, 복잡한 기능 없이 간단히 텍스트 작업을 할 수 있습니다.
            \item 다양한 파일 형식 지원: 텍스트, 코드 파일, 설정 파일 등 다양한 형식의 파일을 편집할 수 있습니다.
            \item 어디서나 사용 가능: 대부분의 운영 체제에서 기본 제공되며, 무료로 사용할 수 있는 다양한 프로그램이 있습니다.
        \end{itemize}
        \item 주요 이점
        \begin{itemize}[<+->]
            \item 직관적 사용법: 복잡한 워드 프로세서보다 직관적이며, 배우기 쉽습니다.
            \item 간단한 작업에 최적화: 빠르게 메모를 하거나, 간단한 설정 파일을 수정할 때 유용합니다.
            \item 프로그래밍의 첫 걸음: 코드를 작성하고 저장하는 첫 번째 단계로 많이 사용됩니다.
        \end{itemize}
        \item 추천 프로그램
        \begin{itemize}[<+->]
            \item Sublime Text + Stata Enhanced
            \item Atom + Stata Kernel
            \item VSCode + VSCode-Stata-Enhanced 또는 Stata Automation
        \end{itemize}
    \end{itemize}
\end{frame}


\begin{frame}[ fragile]
\frametitle{폴더설정}
    \begin{verbatim}
    pwd

    creturn list
        di c(os)
        di c(sysdir_stata)

    cd ~
    cd
    \end{verbatim}
\end{frame}

\section{코딩의 기초}

\begin{frame}[allowframebreaks]
    \frametitle{Stata의 연산자}
    \begin{itemize}[<+->]
        \item 계산
        \begin{itemize}[<+->]
            \item 더하기: $+$
            \item 빼기: $-$
            \item 곱하기: $*$
            \item 나누기: $/$
            \item 승: \textasciicircum\relax
            \item 부정: $-$
            \item 문자연결: $+$
        \end{itemize}
        \framebreak\relax
        \item 논리
        \begin{itemize}[<+->]
            \item AND:\@ \&
            \item OR:\@ $|$
        \item NOT:\@! or \textasciitilde\relax
        \end{itemize}
        \item 관계
        \begin{itemize}[<+->]
            \item grater than: $>$
            \item less than: $<$
            \item no more: $<=$
            \item no less: $>=$
            \item equal: $==$
            \item not equal: $!=$
        \end{itemize}
    \end{itemize}
\end{frame}

\begin{frame}[allowframebreaks, fragile]
    \frametitle{연산자 해석}
    \begin{itemize}[<+->]
        \item 컴퓨터는 조건문에 대하여 참이면 1 거짓이면 0을 반환함.
        \item 컴퓨터는 조건문에 대하여 1 보다 큰 값이면 참으로 간주한다.
        \item AND는 $\times$ OR은 $+$, NOT은 $\times -1$로 간주하면 편하다.
    \end{itemize}
    \begin{exampleblock}{연산자 계산}
        김수현은 대한민국의 35세 남성이라고 하자. 그는 대한민국 남성 또는 30대라는 조건에 포함된다. 조건문으로 해석하면 \[
            male == 1 \& nation == Korea | age == 35
        \]로 표현할 수 있다. 연산 결과는 1 $\times$ 1 $+$ 1 = 2 가 되어 참이다.\\
        김수현은 대한민국 20대 남성이라는 조건에는 벗어난다. 이를 조건문으로 해석하면 \[
            male == 1 \& nation == Korea \& inrange(age, 20, 29)
        \]로 표현할 수 있다. 연산 결과는 1 $\times$ 1 $\times$ 0 = 0 이 되어 거짓이다.
    \end{exampleblock}
\end{frame}

\begin{frame}[allowframebreaks, fragile]
    \frametitle{코멘트}
    \begin{itemize}[<+->]
        \item 모든 코딩은 코멘트가 매우 중요함
        \begin{itemize}[<+->]
            \item 코딩격언: 니가 짠 코딩은 오직 신과 너만이 안다. 이제는 오직 신만이 안다.
            \item 코드생성 및 디버깅 중에 일부 코드를 실행에서 누락 시켜야.
        \end{itemize}
    \item 코멘트의 종류
        \item 명령줄 뒤
        \item 명령줄 전체
        \item 블록: 명령줄 여럿
    \end{itemize}
    \begin{verbatim}
    sum mpg  // This summarizes the mpg variable

    * sum mpg

    /* 
    This is a block comment.
    It spans multiple lines.
    */
    \end{verbatim}
\end{frame}

\begin{frame}[allowframebreaks]
    \frametitle{들여쓰기}
    \begin{itemize}[<+->]
        \item 들여쓰기 (indent)는 코드 가독성의 핵심
        \item 일부 언어는 들여쓰기에 민감하게 반응하기도 함. 대표적으로 python.
    \end{itemize}
\end{frame}


\section{자료가공하기}

\begin{frame}[allowframebreaks, fragile]
    \frametitle{drop, keep}
    \begin{itemize}[<+->]
        \item drop은 참인 대상을 데이터에서 drop
        \item keep은 참인 대상을 남기고 나머지를 데이터에서 drop
        \item keep과 drop은 서로 반대개념
    \end{itemize}
    \begin{verbatim}
        cmdlog using mylogfile, replace
        sysuse auto
        sum
        preserve
        drop price
        sum
        restore , preserve
        drop if price >= 5000
        sum
        restore , preserve
        drop in 40/l
        sum
        restore , preserve
    \end{verbatim}
\end{frame}

\begin{frame}[fragile]
    \frametitle{[], \_n \& \_N}
       \begin{verbatim}
            di price[3]
            gen id = _n
            gen nobs = _N
            bys rep78: gen repid = _n
            bys foreign: gen fobs = _N

            gen lagp = price[_n-1]
            gen leadp = price[_n+1]
            restore , preserve
        \end{verbatim}
\end{frame}

\begin{frame}[allowframebreaks]
    \frametitle{isid \& duplicates}
    \begin{itemize}[<+->]
        \item isid: 해당 변수 또는 변수조합이 id인지 알려주는 명령어.
        \item duplicates: 만약 id인데 isid를 통과 못한다면 어떤식의 문제가 있는지 살펴보는 명령어.
    \end{itemize}
\end{frame}


\begin{frame}[allowframebreaks, fragile]
    \frametitle{reshape}
    \begin{itemize}[<+->]
        \item 자료유형은 크게 time을 자료구조에 어떻게 반영했는가에 의해 long, wide 두 가지 유형이 존재
        \item 시간변수가 별도로 존재하면 long타입.
        \item gdp98 gdp99 같이 다른 시점의 동일정보를 다른 변수로 취급하면 wide 타입.
        \item \textbf{패널자료는 long 타입이 기본임}.
    \end{itemize}
    \begin{verbatim}
        h reshape
        gen year = runiform()
        replace year = 2000 + round(year*100/2,1)
        reshape wide p-g , i(make) j(year)
        reshape long
    \end{verbatim}
\end{frame}

%------------------------------------------------
\end{document}
%------------------------------------------------

