\begin{tabular}{p{1.6cm}|p{4.4cm}|p{8cm}}
\toprule
\textbf{기관명} & \textbf{주요 연혁} & \textbf{주요 업무 및 조직} \\ \midrule
\textbf{고용노동부} & 
1963년 노동청 \newline
1981년 노동부 \newline
2010년 고용노동부 & 
주요업무: 노사관계, 근로기준, 산업안전보건, 고용정책, 고용서비스, 청년여성고용, 직업능력정책 및 국제협력 등 \newline
지방조직: 6개 지방고용노동청, 40개 지청, 고용센터 \newline
산하기관: 근로복지공단, 산업인력공단, 노사발전재단 등 11개 기관 \\ \hline

\textbf{노동위원회} & 
1953년 노동위원회법에 따라 설치 & 
성격: 노동관계의 안정과 발전을 위해 조정과 판정 업무를 독립적 수행, 준사법적 기관 \newline
주요업무: 부당해고 판정, 부당노동행위 및 부당노동행위에 대한 심판, 차별시정, 구제명령 \newline
조직: 중앙노동위원회, 12개 지방노동위원회 및 특별노동위 \\ \hline

\textbf{경제사회발전} \newline \textbf{노사정위원회} & 
1998년 노사정위원회 \newline
2007년 경제사회발전 노사정위원회 \newline
2018년 경제사회노동위원회 & 
기능: 노동정책 및 이와 관련된 경제 사회 정책 통합 논의 \newline
구성: 근로자, 사용자, 정부, 공익을 대표하는 위원으로 구성 \newline
조직: 상무위원회, 의제별 업종별위원회, 사무처, 노사정대표자회의 \\ \bottomrule

\end{tabular}
