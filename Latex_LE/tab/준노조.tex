\begin{tabular}{p{3cm} p{6cm} p{9cm}}
        \toprule
        \textbf{구분} & \textbf{노동조합} & \textbf{준노조} \\
        \midrule
        \textbf{지지층} & 
        경제적인 정체성(계층, 직업, 산업) \newline 핵심 노동자에 집중 &
        정치적·사회적인 정체성 \newline 주변부 노동자에 집중 \\
        \midrule
        \textbf{이해관계 대변} & 
        공통 규칙을 통해 추구하는 집단의 이익, 작업장·기업 수준의 이해관계 &
        유연한 수단을 통해 추구하는 개인과 집단의 다양한 이익, 노동시장 수준의 이해관계 \\
        \midrule
        \textbf{노동자와의 관계} & 
        공식적으로 조직된 노동자 조직 \newline 작업장·기업 수준의 조직화 &
        비조직화된 노동자 \newline 노동시장 수준의 조직화 \\
        \midrule
        \textbf{사용자와의 관계} & 
        단체교섭 활용 & 
        법과 제도 활용 \\
        \midrule
        \textbf{정부와의 관계} & 
        정당정치 활용, \newline 집단 노동관계법 개혁에 중점 &
        정치적 압력 행사, \newline 개별 근로관계법 개혁에 초점 \\
        \bottomrule
    \end{tabular}
