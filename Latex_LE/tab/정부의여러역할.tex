\begin{tabular}{p{4cm}|p{8cm}}  % 열 너비 조정
\toprule
\textbf{역할} & \textbf{내용} \\ \midrule
\textbf{사용자 역할} & 정부는 공무원과 공공부문 근로자에 대한 사용자 역할 수행 \newline 공공부문의 비중이 커지는 상황에서 정부의 사용자 역할 중요 \\ \hline
\textbf{집단적 고용관계의 절차와 게임의 법칙 결정 역할} & 단체협상, 단체행동 등 집단적 고용관계 전반에 대한 원칙과 절차를 정립하는 역할 수행 \\ \hline
\textbf{개별고용관계에 대한 근로기준 설정 역할} & 인간 삶을 영위하는 데 필요한 최소 조건 (최저임금, 사회보장, 보험 등) 보호를 위해 법제화를 하거나 지원방안 마련 \newline 근로자가 최소한의 근로기준과 복지혜택을 보도둑 하는 역할 \\ \hline
\textbf{거시 경제적 역할} & 노동시장의 수요·공급 조정과 노동시장 안정 도모 및 인력의 취업 역량 함양 \newline 정부는 재정·금융정책 및 중앙은행의 금리정책 등을 적절하게 운영하여 노동시장의 안정 도모, 인력의 취업역량 개발 \\ \bottomrule
\end{tabular}
