\begin{tabular}{p{3cm} p{6cm} p{9cm}}
        \toprule
        \textbf{개념} & \textbf{개념} & \textbf{특징} \\
        \midrule
        \textbf{철학적 무노조}
            & CEO의 인재경영에 대한 철학을 갖고 있으며 무노조 경영은 목표가 아닌 경영철학의 부산물 
            & - 우수한 인적자원관리제도가 노조의 존재를 대체 \newline - 무노조프리미엄 \\
            \midrule
        \textbf{정책적 무노조}
            & 무노조경영을 인적자원관리의 목표로 천명하고 노조회피전략을 명시적으로 수립 실행
            & - 직원들을 잘 대우해줘 직무만족도를 향상시켜 노조발생을 억제하는 고진로 (high road)식 방식과 노조결정을 사전에 간파하여 노조결정움직임을 수단과 방법을 가리지 않고 탄압하는 저진로 (low road)식 방식이 있음 \newline - 노조의 대안으로 노사협의회 등과 같은 무노조직원대표조직을 적극 활용 \\
            \midrule
        \textbf{종교적 무노조}
            & 종교적인 믿음을 바탕으로 노사가 합심하여 기업을 운영 
            & 경영자는 노조가 경영자와 직원간의 종교적 화합을 저해하는 불필요한 제삼자로 인식하므로 노조의 결성을 명시적 묵시적으로 금지 \\
            \midrule
        \textbf{영세 무노조}
            & 노조도 없으며 우수한 인적자원관리제도도 없는 기업군 
            & 경영에 대한 전문지식이 부족한 대부분의 중소영세기업에서 나타남. 블랙홀 또는 블리크 하우스라고도 함 \\
            \bottomrule
    \end{tabular}
