\documentclass[11pt,answers]{exam} %정답지를 만들경우 answers 옵션을 넣을 것.
%----------------------------------------------------------
% PACKAGES AND THEMES
%----------------------------------------------------------
\usepackage{kotex}
\usepackage{lastpage, setspace, amsmath}

% "point"와 "points"를 "점"으로 변경
    \renewcommand{\pointname}{점}
    \renewcommand{\points}{\pointname}
% 줄 간격 및 문단 간격 조정
    \setlength{\parskip}{0pt}  % 문단 간 여백 제거
    \setstretch{1.2}  % 줄 간격 조정
% 선택지 번호를 숫자로 변경
    \renewcommand{\thechoice}{\arabic{choice}}
% 선택지 번호를 한글 자모로 변경하는 매크로 (ㄱ, ㄴ, ㄷ, ㄹ)
    %\newcommand{\koreanChoice}{
        %\ifcase\numexpr\value{choice}-1\relax ㄱ\or\relax ㄴ\or\relax ㄷ\or\relax
        %ㄹ\or\relax ㅁ\or\relax ㅂ\or\relax ㅅ\or\relax ㅇ\or\relax ㅈ\or\relax
        %ㅊ\or\relax ㅋ\or\relax ㅌ\or\relax ㅍ\or\relax ㅎ\fi
    %}
    %\renewcommand{\thechoice}{\koreanChoice}
% part 문제 번호를 한글 자모(ㄱ, ㄴ, ㄷ)로 변경하는 매크로
    \newcommand{\koreanPart}{
        \ifcase\value{part}
            ㄱ\or\relax ㄴ\or\relax ㄷ\or\relax ㄹ\or\relax ㅁ\or\relax
            ㅂ\or\relax ㅅ\or\relax ㅇ\or\relax ㅈ\or\relax
            ㅊ\or\relax ㅋ\or\relax ㅌ\or\relax ㅍ\or\relax ㅎ\fi
    }
    \renewcommand{\thepart}{\koreanPart}
    \renewcommand{\partlabel}{\thepart.}
% 페이지 번호를 하단에 출력
    \firstpagefooter{}{\thepage/\pageref{LastPage} 쪽}{}
    \runningfooter{}{\thepage/\pageref{LastPage} 쪽}{}
%----------------------------------------------------------
\begin{document}
%----------------------------------------------------------

% TITLE
    \title{\relax
        2024년 2학기 기말고사 \\
        \Large
        노사관계의 이론과 실재
    }
    \author{담당교수: 오성재}
    \date{2024년 12월 18일}
    \maketitle

% 학생 이름과 학번 입력란
    \noindent
    성명: \makebox[.3\textwidth]{\hrulefill} \\[3pt]
    학번: \makebox[.3\textwidth]{\hrulefill}

% 점수 추적을 시작
    \addpoints\relax
    %\begin{center}
        %\gradetable[h][questions] 
    %\end{center}

\section*{객관식 문제: (문제당 2점)}

\begin{questions}

\question임금에 대한 설명으로 옳지 않은 것은?
    \begin{choices}
    \choice\relax 임금수준의 상한선은 기업이 지불할 수 있는 능력에 달려 있다.
    \choice\relax 초임수준이 낮고 기술진보의 속도가 급격하지 않았던 시기에는 연공급이 바람직했다. 
    \CorrectChoice\relax 직능급은 현재 담당하고 있는 일의 가치를 평가하여 임금을 지급한다
    \choice\relax 성과급은 업무간 상호의존성이 높은 경우에는 시행이 어렵다. 
    \end{choices}

\question무노조기업에 관한 설명으로 옳지 않은 것은? 
    \begin{choices}
    \choice\relax 개인 차원의 고용관련 보호법안 통과로 집단노사관계의 필요성이 감소하여 무노조기업이 증가해왔다.
    \choice\relax 철학적 무노조 기업은 직원들의 처우를 좋게 하고 의사소통 채널을 둠으로써 노조 결성의 필요성과 의욕을 감소시킨다.  
    \CorrectChoice\relax 정책적 무노조기업은 근로자의 직무만족도를 높이기 위한 인적자원관리제도를 활용하는 대신 수단과 방법을 가리지 않고 노조를 회피하는 전략을 사용한다 
    \choice\relax 사용자의 지불능력이 미약한 한계기업은 노동조합을 결성하여도 처우 향상의 보장이 없으므로 무노조 기업이 된다.  
    \end{choices}

\question노사협의회 구성원의 역할과 관련하여 올바른 설명은?
    \begin{choices}
    \choice\relax 노동자 대표만이 의결권을 갖는다.
    \choice\relax 경영진 대표는 의결권이 없으며, 오직 자문 역할만 수행한다.
    \CorrectChoice\relax 노사 양측 대표는 의결권을 공유하며, 협의회의 결정에 동등하게 참여한다.
    \choice\relax 경영진 대표만이 최종 결정권을 갖는다.
    \end{choices}

\question우리나라 공공부문에 대한 설명으로 옳지 않은 것은?
    \begin{choices}
    \choice\relax 높은 노동조합 조직률로 인해 미래 한국의 고용관계의 중심축은 민간부문에서 공공부문으로 옮겨갈 가능성이 크다.
    \CorrectChoice\relax 우리나라 공무원 노조는 단체교섭은 가능하지만 단체협약을 체결할 권리가 없다
    \choice\relax 공공부문의 노동권 보장을 반대하는 사람들은 공공부문에서는 노사가 유착하여 과다한 임금인상과 근로조건 개선을 도모할 것이라고 주장한다. 
    \choice\relax 1999년 『교원의 노동조합 설립 및 운영에 관한 법률』제정으로 합법화된 전국교직원노동조합은 해고자를 조합원으로 받아들였다는 이유로 법외노조 판정을 받았다. 
    \end{choices}

\question다음 중 옳은 것은?
    \begin{choices}
    \CorrectChoice\relax 평균임금에 기업의 사회보험료를 합한 금액을 보수비용이라 한다.
    \choice\relax 산업재해보상금은 통상임금에 근거하여 산정한다.
    \choice\relax 연장근로 수당은 평균임금의 50/100 이상을 가산하여 지급한다.
    \choice\relax 사용자의 잘못으로 근로자가 일을 쉬게 되었을 경우에 사용자는 임금을 전액 지급해야 한다. 
    \end{choices}

\question다음 중 옳지 않은 것은?
    \begin{choices}
    \choice\relax 최저임금을 결정하기 위한 생계비를 실제생계비 방법으로 조사하면 소득 충격에 따른 가계지출 변동으로 생계비가 과소추정되거나 과대추정되는 문제가 있다. 
    \CorrectChoice\relax 퇴직금 제도를 퇴직연금 가입 의무화로 바꾼 것은 영세 중소기업 근로자보다 대기업 근로자에게 안정적인 노후대책 수단이 될 가능성을 제고했다.  
    \choice\relax 기업은 비슷한 자격이 있는 직원에게 비슷한 수준의 임금을 지급하여 직원들이 불만을 갖지 않도록 임금을 관리한다.
    \choice\relax 300인 이상 사업장은 미만인 사업장보다 연공급 임금체계를 가지는 비중이 훨씬 높다. 
    \end{choices}

\question다음중 ILO 비준장려협약 가운데 핵심협약이 아닌 것은?
    \begin{choices}
    \choice\relax 차별금지관련
    \CorrectChoice\relax 산업안전보건
    \choice\relax 강제근로관련
    \choice\relax 아동근로
    \end{choices}

\question다음 중 옳지 않은 것은?
    \begin{choices}
    \choice\relax 교원은 통일교섭만 허용된다.
    \choice\relax 교원의 노사관계에서 사용자측은 단체교섭 시 교섭을 위임할 수 없다.
    \choice\relax 교원노동조합은 1960년 4․19혁명 이후 처음 결성되었다. 
    \CorrectChoice\relax 대학교수의 노동조합 가입을 불허하고 있다.
    \end{choices}

\question다음중 해고와 관련된 설명으로 틀린 것은?
    \begin{choices}
    \CorrectChoice\relax 최소 3개월 이전에 해고를 예고해야 한다.
    \choice\relax 긴박한 경영상의 필요가 인정되어야 한다.
    \choice\relax 사용자는 해고를 피하기위한 노력을 다해야 한다.
    \choice\relax 합리적이고 공정한 해고의 기준을 정하고 이에 따라 그 대상자를 선정해야 한다.
    \end{choices}
\section*{빈 칸 채우기: (빈 칸당 2점)}

\question\fillin[스톡옵션]{}을/를 도입하는 취지는 유동성이 부족한 중소기업이 유능한 인재를 유치할 수 있도록 하기 위한 것이다.  
\question우리나라는 2003년 외국인근로자의 \fillin[고용허가제]{}을/를 도입한 이후 외국인 근로자와 내국인 근로자를 동등하게 대우하고 있다. 
\question파견근로자보호법은 파견근로기간은 원칙적으로 \fillin[1]{}년을 초과하지 못하도록 규정하고 있다. 
\question사용자는 신생노조의 결성 또는 기존 노동조합 활동을 방해하기 위해 \fillin[노조파괴전문가]{}을/를 활용한다.
\question\fillin[성과배분]{}은/는 피고용인이 기업의 경영에 참가하여 성과를 향상시키기 위해 노력하여 발생한 이익을 금전적 형태로 피고용인에게 배분하는 제도이다. 
\question\fillin[비노조]{}은/는 사용자가 노조 없이 경영하도록 한 선택의 결과를 의미하며 전략적으로 노조를 배제하는 정책을 펴는 기업에서 사용하는 용어이다. 
\question\fillin[통상임금]{}은/는 매월정기적, 일률적, 고정적으로 지급되는 급여를 의미한다. 여기에 각종 수당을 더해 월 급여로 받는 모든 금액을 \fillin[평균임금]{}이라고 한다.
\question우리나라는 ILO 핵심협약 가운데 105호를 비준하지 않고 있다. 이는 \fillin[정치적 견해표명 등]{}에 의한 처벌로 강제노동을 금지하는 것이다.
\question한국, 미국, 일본, 영국을 ILO 협약 비준이 많은 순서대로 나열하면 \fillin[영국, 일본, 한국, 미국]{} 이다. 
\pagebreak

\section*{약술형 문제}
모든 약술형 답은 \textbf{수업시간에 배운 개념}을 바탕으로 답과 근거를 모두 쓰시오.

\question조별활동과 관련하여 다음의 질문에 답하시오.
\begin{parts}
    \part[10] 자기 조에서 발표한 사례에 대한 분석 및 시사점을 요약하시오
        \ifprintanswers\relax
            \begin{itemize}
                \item \textbf{설명}: 가나다라마바사
                \item \textbf{부분점수}: 
                    \begin{itemize}
                        \item 가나다라마바사
                    \end{itemize}
            \end{itemize}
        \else 
            \\[3pt]
            \rule{\linewidth}{0.4pt} \\[3pt]
            \rule{\linewidth}{0.4pt} \\[3pt]
            \rule{\linewidth}{0.4pt} \\[3pt]
            \rule{\linewidth}{0.4pt} \\[3pt]
            \rule{\linewidth}{0.4pt} \\[3pt]
            \rule{\linewidth}{0.4pt} \\[3pt]
            \rule{\linewidth}{0.4pt} \\[3pt]
            \rule{\linewidth}{0.4pt}
        \fi
    \stepcounter{part}
    \part[15] 자기 조에서 논평한 사례에 대한 제언을 쓰시오. (발표당시 제언이 부족하다는 지적을 받은 경우, 논평을 그대로 쓴다면 낮은 점수를 받을 것임.)
        \ifprintanswers\relax
            \begin{itemize}
                \item \textbf{설명}: 가나다라마바사
                \item \textbf{부분점수}: 가나다라마바사
            \end{itemize}
        \else
            \\[3pt]
            \rule{\linewidth}{0.4pt} \\[3pt]
            \rule{\linewidth}{0.4pt} \\[3pt]
            \rule{\linewidth}{0.4pt} \\[3pt]
            \rule{\linewidth}{0.4pt} \\[3pt]
            \rule{\linewidth}{0.4pt} \\[3pt]
            \rule{\linewidth}{0.4pt} \\[3pt]
            \rule{\linewidth}{0.4pt} \\[3pt]
            \rule{\linewidth}{0.4pt}
        \fi
\end{parts}

\question[20]  노조조직률 하락은 선진국에서 나타나는 공통적인 현상이다. 현상의 원인에 대해 설명하시오.
    \ifprintanswers\relax
        \begin{itemize}
            \item \textbf{설명}: 가나다라마바사
            \item \textbf{부분점수}: 가나다라마바사
        \end{itemize}
    \else
        \\[3pt]
        \rule{\linewidth}{0.4pt} \\[3pt]
        \rule{\linewidth}{0.4pt} \\[3pt]
        \rule{\linewidth}{0.4pt} \\[3pt]
        \rule{\linewidth}{0.4pt} \\[3pt]
        \rule{\linewidth}{0.4pt} \\[3pt]
        \rule{\linewidth}{0.4pt} \\[3pt]
        \rule{\linewidth}{0.4pt} \\[3pt]
        \rule{\linewidth}{0.4pt}
    \fi
\question[10] 최근 일어난 대통령 탄핵과 관련하여, 정의의 관점 가운데 하나를 택하여 의견을 제시하시오. (찬반 여부는 채점하지 않고, 오로지 자신의 입장을 정당화 하는 과정에 대해서만 채점함.)
    \ifprintanswers\relax
        \begin{itemize}
            \item \textbf{설명}: 가나다라마바사
            \item \textbf{부분점수}: 가나다라마바사
        \end{itemize}
    \else
        \\[3pt]
        \rule{\linewidth}{0.4pt} \\[3pt]
        \rule{\linewidth}{0.4pt} \\[3pt]
        \rule{\linewidth}{0.4pt} \\[3pt]
        \rule{\linewidth}{0.4pt} \\[3pt]
        \rule{\linewidth}{0.4pt} \\[3pt]
        \rule{\linewidth}{0.4pt} \\[3pt]
        \rule{\linewidth}{0.4pt} \\[3pt]
        \rule{\linewidth}{0.4pt}
    \fi

\question[10] 중대재해처벌법의 영향과 시사점을 서술하시오.
    \ifprintanswers\relax
        \begin{itemize}
            \item \textbf{설명}: 가나다라마바사
            \item \textbf{부분점수}: 가나다라마바사
        \end{itemize}
    \else
        \\[3pt]
        \rule{\linewidth}{0.4pt} \\[3pt]
        \rule{\linewidth}{0.4pt} \\[3pt]
        \rule{\linewidth}{0.4pt} \\[3pt]
        \rule{\linewidth}{0.4pt} \\[3pt]
        \rule{\linewidth}{0.4pt} \\[3pt]
        \rule{\linewidth}{0.4pt} \\[3pt]
        \rule{\linewidth}{0.4pt} \\[3pt]
        \rule{\linewidth}{0.4pt}
    \fi

\end{questions}
%------------------------------------------------
\end{document}
%------------------------------------------------
