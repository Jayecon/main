%----------------------------------------------------------
%	PACKAGES AND THEMES
%----------------------------------------------------------

\documentclass[aspectratio=169,xcolor=dvipsnames,handout]{beamer}
\usetheme{Darmstadt}
\usecolortheme{seahorse}
\usepackage[hangul]{kotex}
\usepackage{hyperref}
\usepackage{graphicx, array, adjustbox}
\usepackage[table]{xcolor}
\usepackage{booktabs, multicol, multirow}
\setbeamercovered{transparent}

\definecolor{myblue}{RGB}{184, 185, 226}
%----------------------------------------------------------
%	TITLE PAGE
%----------------------------------------------------------

\title[들어가는 말]{노사관계론 관련 배경지식} % The short title appears at the bottom of every slide, the full title is only on the title page
\subtitle{노사관계론}
\author[오성재]{오성재}
\institute[CNU] % Your institution as it will appear on the bottom of every slide, may be shorthand to save space
{\relax
    충남대학교 경제학과\\
}
%\date{\today} 
\date{2024년 9월 2일} 

\begin{document}

\begin{frame}
    \titlepage\relax
\end{frame}

\begin{frame}{목차}
    \tableofcontents
\end{frame}

\begin{frame}{들어가는 말}
    \begin{itemize}
        \item 본격적인 수업의 진행에 앞서 두 가지 배경지식을 습득.
        \begin{itemize}
            \item 노동과 근로소득은 불가분의 관계이고, 따라서 소득에 대한 이해는 학습에 도움이 됨.
            \item 수업의 특성상 노동관련 법이 자주 등장하므로 배경지식으로 짚고 넘어갈 필요가 있음.
        \end{itemize}
    \end{itemize}
\end{frame}
%------------------------------------------------
\section{한국의 소득}
%------------------------------------------------

\begin{frame}{가계금융복지조사}
    \begin{itemize}
        \item 통계청 주관으로 2012년부터 전국 2만 가구를 표본으로 매년 실시되는 조사.
        \item 가구단위의 자산, 부채, 소득, 지출 등을 조사.
        \begin{itemize}
            \item 가구: 1인 또는 2인 이상이 모여 주거 또는 소득과 지출 등 생계를 같이 하는 생활 단위.
        \end{itemize}
        \item 각종 정책의 기준으로 매우 중요한 자료.
        \begin{itemize}
            \item 중산층의 기준: 중위소득의 50--150\%.
            \item 소득분위: 국가장학금 신청의 기준.
        \end{itemize}
    \end{itemize}
\end{frame}

\begin{frame}{가구소득의 구성}
    \begin{itemize}
        \item 근로소득.
        \item 사업소득.
        \item 재산소득: 금융소득, 임대소득.
        \item 공적이전소득.
        \begin{itemize}
            \item 연금: 공적연금, 기초연금.
            \item 수당: 양육수당, 장애수당.
        \end{itemize}
        \item 사적이전소득
        \begin{itemize}
            \item 용돈, 상속, 유산 등등. 
        \end{itemize}
    \end{itemize}
\end{frame}

\begin{frame}{가구소득 분포}
\centering
\begin{figure}
    \includegraphics[width=\textwidth]{pic/가구소득 구간별 가구분포.png}
    \caption{가구소득 구간별 가구분포}
\end{figure}

\end{frame}
\begin{frame}{소득원천별 가구소득}
\begin{columns}
    \begin{column}{.5\textwidth}
        \begin{figure}
            \centering
            \includegraphics[width=.8\textwidth]{pic/소득원천별 가구소득 평균.png}
            \caption{소득원천별 가구소득 평균}
        \end{figure}
    \end{column}    
    \begin{column}{.5\textwidth}
        \begin{figure}
            \centering
            \includegraphics[width=.8\textwidth]{pic/소득원천별 가구소득 구성비.png}
            \caption{소득원천별 가구소득 구성비}
        \end{figure}
    \end{column}    
\end{columns}
\end{frame}

\begin{frame}{가구특성 및 가구소득 구간별 가구분포}
    \begin{table}
        \raggedleft\relax
        \tiny{(단위: \%)}
        \hspace*{2em}
        \\
        \centering
        \resizebox{.95\textwidth}{!}{\relax
           \begin{tabular}{c|c|c|c|c|c|c|c|c}
\toprule
\multicolumn{2}{c|}{} &\textbf{전체} & \textbf{1천만원 미만} & \textbf{1-3천만원 미만} & \textbf{3-5천만원 미만} & \textbf{5-7천만원 미만} & \textbf{7-10천만원 미만} & \textbf{1억원 이상} \\ \midrule
\multicolumn{2}{c|}{\textbf{2021년 전체}} & 100.0 & 5.8 & 22.8 & 20.4 & 16.3 & 16.6 & 18.0 \\ \hline
\multicolumn{2}{c|}{\textbf{2022년 전체}} & 100.0 & 5.2 & 21.6 & 19.8 & 16.4 & 17.0 & 20.0 \\ \hline
\multirow{4}{*}{\textbf{가구주 연령대}} & 39세 이하 & 100.0 & 1.9 & 17.7 & 25.5 & 19.4 & 18.9 & 16.6 \\ \cline{3-9}
 & 40-49세 & 100.0 & 1.7 & 10.0 & 18.0 & 19.7 & 22.8 & \cellcolor{myblue}27.9 \\ \cline{3-9}
 & 50-59세 & 100.0 & 2.7 & 13.8 & 16.9 & 16.2 & 19.9 & \cellcolor{myblue}30.5 \\ \cline{3-9}
 & 60세 이상 & 100.0 & \cellcolor{myblue}10.1 & 34.1 & 19.8 & 13.5 & 11.4 & 11.1 \\ \hline
\multirow{4}{*}{\textbf{가구주 종사상 지위}} & 상용근로자 & 100.0 & 0.6 & 9.2 & 19.0 & 19.1 & 22.3 & 29.8 \\ \cline{3-9}
 & 자영업자 & 100.0 & 0.9 & 15.6 & 23.3 & 20.2 & 19.7 & 20.3 \\ \cline{3-9}
 & 임시·일용근로자 & 100.0 & 5.9 & 41.8 & 22.9 & 12.9 & 10.5 & 6.0 \\ \cline{3-9}
 & 기타(무직 등) & 100.0 & 19.2 & 43.3 & 16.1 & 8.7 & 6.4 & 6.3 \\ \bottomrule
\end{tabular}
        }
       \caption{가구특성 및 가구소득 구간별 가구분포} 
    \end{table}
\end{frame}

\begin{frame}{소득원천별 가구소득 평균 및 구성비}
    \begin{table}
        \raggedleft\relax
        \tiny{(단위: 만원, \%, \%p)}
        \hspace*{2em}
        \\
        \centering
        \resizebox{.95\textwidth}{!}{\relax
           \begin{tabular}{c|c|c|c|c|c|c|c|c}
\toprule
\multicolumn{2}{c|}{} & \multicolumn{6}{c|}{\textbf{평균}} & \multirow{2}{*}{\textbf{중앙값}} \\ \cline{3-8}
\multicolumn{2}{c|}{} & \textbf{가구소득} & \textbf{근로소득} & \textbf{사업소득} & \textbf{재산소득} & \textbf{공적이전소득} & \textbf{사적이전소득} & \\ \midrule \midrule
\multicolumn{2}{c|}{\textbf{2021년}} & 6,470 & 4,125 & 1,160 & 426 & 656 & 103 & 5,098 \\ 
\multicolumn{2}{c|}{\textbf{2022년}} & 6,762 & 4,390 & 1,206 & 436 & 625 & 106 & 5,362 \\ \hline
\multicolumn{2}{c|}{\textbf{증감}}   & 293 & 265 & 46 & 11 & -32 & 3 & 264 \\ \hline
\multicolumn{2}{c|}{\textbf{증감률}} & 4.5 & 6.4 & 4.0 & 2.5 & -4.8 & 2.7 & 5.2 \\ \hline
\multirow{3}{*}{\textbf{구성비}} & \textbf{2021년} & 100.0 & \cellcolor{myblue}63.8 & 17.9 & 6.6 & 10.1 & 1.6 & - \\
 & \textbf{2022년} & 100.0 & \cellcolor{myblue}64.9 & 17.8 & 6.4 & 9.2 & 1.6 & - \\ \cline{2-9}
 & \textbf{전년차}& - & 1.2 & -0.1 & -0.1 & -0.9 & 0.0 & - \\ \bottomrule
\end{tabular}
        }
       \caption{소득원천별 가구소득 평균 및 구성비}
    \end{table}
\end{frame}

\begin{frame}{소득원천별 가구소득 평균 및 구성비 상세}
    \begin{table}
        \raggedleft\relax
        \tiny{(단위: 만원, \%)}
        \hspace*{2em}
        \\
        \centering
        \resizebox{.95\textwidth}{!}{\relax
           \begin{tabular}{c|c|c|c|c|c|c|c}
\toprule
\multicolumn{2}{c|}{} & \textbf{가구소득} & \textbf{근로소득} & \textbf{사업소득} & \textbf{재산소득} & \textbf{공적이전소득} & \textbf{사적이전소득} \\ \midrule 
\multicolumn{2}{c|}{\textbf{2022년 전체}} & 6,762 & 4,390 & 1,206 & 436 & 625 & 106 \\ \hline
\multirow{5}{*}{\textbf{평균}} & 1분위 & 1,405 & 394 & 106 & 98 & 611 & 195 \\
 & 2분위 & 3,309 & 1,755 & 571 & 183 & 645 & 155 \\ 
 & 3분위 & 5,388 & 3,265 & 1,074 & 284 & 685 & 80 \\
 & 4분위 & 8,111 & 5,519 & 1,541 & 375 & 616 & 60 \\
 & 5분위 & 15,598 & 11,015 & 2,737 & 1,240 & 566 & 39 \\ \hline
\multirow{5}{*}{\textbf{소득원천별 구성비}} & 전체 & 100.0 & 64.9 & 17.8 & 6.4 & 9.2 & 1.6 \\ \cline{2-8}
 & 1분위 & 100.0 & \cellcolor{myblue}28.0 & 7.6 & 7.0 & \cellcolor{myblue}43.5 & \cellcolor{myblue}13.9 \\
 & 2분위 & 100.0 & 53.0 & 17.2 & 5.5 & 19.5 & 4.7 \\
 & 3분위 & 100.0 & 60.6 & 19.9 & 5.3 & 12.7 & 1.5 \\
 & 4분위 & 100.0 & 68.0 & 19.0 & 4.6 & 7.6 & 0.7 \\ 
 & 5분위 & 100.0 & 70.6 & 17.6 & 8.0 & 3.6 & 0.3 \\ \bottomrule
\end{tabular}
        }
       \caption{소득원천별 가구소득 평균 및 구성비 상세}
    \end{table}
\end{frame}

\begin{frame}{균등화처분가능소득}
    \begin{itemize}
        \item 균등화소득:  가구원수가 다른 가구 간의 후생 (복지)수준을 비교할 수 있도록 가구소득을 $\sqrt{\text{가구원수}}$ 로 나눈 소득
        \item 시장소득 = 근로소득 + 사업소득 + 재산소득 + 사적 이전소득 – 사적 이전지출.
        \item 처분가능소득 = 시장소득 + 공적 이전소득 - 공적 이전지출.
        \item 분배 및 불평등 관련 소득은 \textbf{균등화처분가능소득}이 기준.
    \end{itemize}
\end{frame}

\begin{frame}{균등화처분가능소득 분포}
\centering
\begin{figure}
    \includegraphics[width=\textwidth]{pic/균등화처분가능소득분포.png}
    \caption{균등화처분가능소득 분포}
\end{figure}
\end{frame}

%------------------------------------------------
\section{한국의 법 체계}
%------------------------------------------------

\begin{frame}{우리법 체계}
    \begin{itemize}
        \item 우리 법은 헌법, 법률, 명령, 조례 및 규칙 순으로 위계를 가짐.
        \item 상위 법은 하위 법을 규정함: 제정, 시행, 처벌 등등.
        \item 하위 법은 상위 법을 어길 수 없음.
    \end{itemize}
\end{frame}

\begin{frame}{헌법}
    \begin{itemize}
        \item 대한민국의 최고 규범으로서, 국가의 기본 구조와 국민의 기본권을 규정.
        \item 헌법 제1조 제2항 ``대한민국의 주권은 국민에게 있고, 모든 권력은 국민으로부터 나온다''
        \item 헌법 개정은 국회 재적의원의 $2/3$ 동의로 발의되고 국민투표에서 과반수를 넘겨야 가능.
        \item 헌법 제32조 근로의 권리.
        \item 헌법 제33조 근로자의 단결권.
    \end{itemize}
\end{frame}

\begin{frame}{법률}
    \begin{itemize}
        \item 국회에서 제정된 법으로, 헌법의 위임을 받아 구체적인 사항을 규율.
        \item 법률은 모든 국민에게 적용되며, 하위 법령인 명령, 규칙 등은 법률을 집행하기 위한 세부 규정.
        \item 헌법 제40조 ``입법권은 국회에 속한다''
        \item 근로기준법, 노조법, 산업안전보건법, 최저임금법, 공무원노조법 등등.
    \end{itemize}
\end{frame}

\begin{frame}{명령}
    \begin{itemize}
        \item 행정부에서 법률을 집행하기 위해 발하는 법령으로 대통령령, 총리령, 부령 등이 존재.
        \item 헌법 제75조 ``대통령은 법률에서 구체적으로 범위를 정하여 위임받은 사항과 법률을 집행하기 위하여 필요한 사항에 관하여 명령을 발할 수 있다''.
        \item 각 법률의 시행령 등등.
    \end{itemize}
\end{frame}

\begin{frame}{조례 및 규칙}
    \begin{itemize}
        \item 조례는 지방자치단체의 의회에서 제정하는 법규, 지방자치단체의 자치권을 행사하기 위한 구체적인 규정. 
        \item 규칙은 지방자치단체장이 법령 또는 조례의 범위 내에서 제정하는 규범.
        \item 헌법 제117조 제1항 ``지방자치단체는 법령의 범위 안에서 그 사무에 관하여 조례를 제정할 수 있다''
    \end{itemize}
\end{frame}
%------------------------------------------------
\end{document}
