%----------------------------------------------------------
% PACKAGES AND THEMES
%----------------------------------------------------------
\documentclass[aspectratio=169,xcolor=dvipsnames,handout]{beamer}

\usetheme{Darmstadt}
\usecolortheme{seahorse}
\setbeamercovered{transparent}

\usepackage[hangul]{kotex}
\usepackage{hyperref}
\usepackage{graphicx, array, adjustbox, makecell}
\usepackage{booktabs, multicol, multirow}

% font조정
%\usepackage{fontspec}
%\setmainfont{Times New Roman}
%\setmainhangulfont{NanumGothic}

% 문자열 대체{노사관계론 전용}
\usepackage{newunicodechar}
\newunicodechar{•}{$\cdot$}
\newunicodechar{➔}{$\implies$}
\newunicodechar{∴}{$\therefore$}
\newunicodechar{∵}{$\because$}

%----------------------------------------------------------
% TITLE PAGE
%----------------------------------------------------------
\title{공공부문 고용관계}
\subtitle{노사관계의 이론과 실제}
\author{오성재}
\institute[CNU]
{\relax
    충남대학교 경제학과\
    }
\date{2024년 11월 11일}

%----------------------------------------------------------
\begin{document}
%----------------------------------------------------------

\frame{\titlepage}

\begin{frame}{목차}
    \small
    \tableofcontents[hideallsubsections]
\end{frame}

\section{배경과 이론}
\begin{frame}
    \frametitle{개관}
    \begin{itemize}[<+->]
        \item 공공부문: 정부 즉 국가 또는 지방자치단체가 사용자 역할을 하는 기관 
        \item 21세기의 한국고용관계의 중심축이 공공부문으로 옮겨갈 가능성이 높아짐 
        \begin{itemize}
            \item 전체 근로자의 노동조합조직률이 약 10.0\%에 머무르고 있는 반면에 공공부문 노조조직률은 약 68\% 수준
            \item 외국 사례 (외국에서는 공공부문의 고용관계가 노동운동의 핵심이 된지 오래됨)
        \end{itemize}
        \item 공공부문 고용관계에서 특히 주목해야 이유: 사용자와 일반시민의 관심사인 공공성과 직원들의 관심사인 노동기본권간의 상충성
    \end{itemize}
\end{frame}

\begin{frame}[allowframebreaks]
    \frametitle{특징}
    \begin{itemize}[<+->]
        \item 사용자 불명료성, 중층성 $\implies$ 공공부문 사용자는 중첩되며, 공공부문 경영자의 권한은 분산되어 있고, 
        \item 사용자를 특정하기 어려움. 
        \begin{itemize}
            \item 정부는 공공부문에 대해 궁극적인 사용자대표성을 가지면서 지휘감독. 그러나 직접 고용관계담당자는 아니라는 점에서 역할과 기능이 모호함. 
        \end{itemize}
        \begin{exampleblock}{공공기관의 사용자는 누구인가?}
            \begin{enumerate}[<+->]
                \item 공공기관의 법률상 사용자는 해당 기관장임.
                \item 그러나 실제 기관운영은 집권한 정부의 정책 방향, 기획재정부의 예산 편성 지침과 예산 집행 지침에 크게 제약을 받음.
                \item 더 나아가 정부 예산은 의회의 승인을 받아야 함.
                \item 각 공공기관은 경영평가와 감사원의 감사대상이 됨.
                \item 최종적으로는 납세자가 공공부문 운영 비용을 담당하고 있음. 
            \end{enumerate}
        \end{exampleblock}
        \item 다면교섭: 셋 이상 당사자가 참여하고, 공식적인 교섭상대방이 아니라 다양한 이해관계자집단과 교섭하고, 경우에 따라 상위기관의 공식, 비공식적인 인준을 필요로 하는 교섭. 
        \begin{itemize}
            \item 공공부문에서 나타나는 다면교섭 (multi-lateral bargaining in public sector)은 민간부문에서 나타나는 노사양자교섭 (bi-lateral bargaining in private sector)과 두드러진 차이를 보임.
        \end{itemize}
    \framebreak%
        \item 노동권과 서비스이용권 상충가능성: 공공부문은 조직 안에서 노사 간 갈등과 함께, 노조와 정부, 노조와 납세자/소비자 간 갈등도 있을 수 있음.  
        \begin{itemize}
            \item 공공의 이익을 보호하기 위하여 공공부문 노동쟁의 해결에 파업권을 제한하고 중재제도 등을 적극적으로 활용.
        \end{itemize}
        \item 노사유착가능성: 공공부문 종사자들은 공공부문 내부에서는 관리자와 직원으로 나누어진다고 해도 정부 또는 납세자에 대해서는 모두 피고용인이기 때문에 관리자와 직원 사이에 이해공동성이 나타날 수 있음.
        \item 노조결성의 상대적 용이성: 공공기관 경영자는 노조를 반대할 특별한 유인이 없음. 한국 경우에도 공공부문 노동조합 조직률은 민간부문보다 훨씬 높음. 이후 노동운동을 공공부문이 주도할 것이라는 예측을 가능하게 함.
    \end{itemize}
\end{frame}

\begin{frame}[allowframebreaks]
    \frametitle{노동기본권을 둘러싼 논쟁}
    \begin{itemize}[<+->]
        \item 공공부문 노동권 보장 논쟁: 1960년대 미국에서 공무원과 교사의 노동조합 결성 과정 논쟁에서 촉발
        \item 노동권 보장 반대 
        \begin{itemize}
            \item 노사 간 유착 가능성 높음
            \begin{itemize}
                \item 과다한 임금인상과 노동조건 개선
                \item 납세자의 감시기능 한계 
            \end{itemize}
            \item 공공부문 노조의 협상력이 큼 (∵ 파업 시 대체수단이 없음)
            \begin{itemize}
                \item 사용자의 과대한 양보가 나타날 수 있음
            \end{itemize}
        \end{itemize}
        \framebreak%
        \item 노동권 보장 찬성 
        \begin{itemize}
            \item 공공부문 노동자도 노동자이므로 노동기본권 보장은 필수
            \item 납세자 집단의 적극적 참여로 노사유착 방지 가능
            \item 노동3권의 일부를 제한하면 노조의 협상력 발휘가 제한됨
        \end{itemize}
        \item 실증연구 결과: 공공부문의 노동기본권 보장이 노사 간 협상력 왜곡을 가져오지 않음. 시민의 공공서비스 향유권과 노동기본권의 적절한 균형이 가능함 
    \end{itemize}
    \begin{table}
        \centering
        \resizebox{.7\textwidth}{!}{\relax
            \begin{tabular}{lp{8cm}p{8cm}}
\toprule
& \textbf{공공부문} & \textbf{민간부문} \\
\midrule
\textbf{주요 목적} & 공공성 추구 & 이윤 추구 \\
\midrule
\textbf{관련 기제} & 정부 & 시장 \\
\midrule
\textbf{예산제약} & 공공적 통제 & 경제적 제약 \\
\midrule
\textbf{정치적 성격} & 정치적 성격 강함 & 정치적 성격 약한 편 \\
\midrule
\textbf{사용자} & 사용자불명확성 \newline 경영자 자율성 약함 \newline 사실상의 사용자로서 정부 존재 & 사용자가 명확함 \newline 경영자 자율성 강함 \newline \\
\midrule
\textbf{통제} & 다양한 이해집단에 의한 다중통제 & 기업내 통제 \\
\midrule
\textbf{경쟁} & 국내외 시장 힘으로부터 비교적 자유로움 & 국내외 기업과 경쟁 \\
\midrule
\textbf{목표} & 목표다양성, 목표간 상호모순 가능성 & 비교적 명확하고 단일한 목표 \\
\midrule
\textbf{노사관계} & 갈등관계 또는 유착관계 & 갈등관계 또는 중속관계 \\
\midrule
\textbf{교섭} & 노조/사용자/정부/시민 다면교섭 & 노조/사용자 양자교섭 \\
\midrule
\textbf{갈등} & 기업내 노사 갈등 & 기업내 노사 갈등 \\
\bottomrule
\end{tabular}


        }
        \caption{부문별 고용관계의 특징 비교}
    \end{table}
\end{frame}

\begin{frame}[allowframebreaks]
    \frametitle{공공부문 고용관계 추세}
    \begin{itemize}[<+->]
        \item 최근 노동운동에서 나타나고 있는 여러 조짐들은 다른 선진국과 마찬가지로 한국고용관계 중심축이 공공부문으로 옮겨가고 있음을 보여 줌.
        \item 부문별 노동조합조직율:  2021년 말 기준 공공부문 노조조직률은 공무원이 75.3\%, 교원이 18.8\%, 공공기관이 70.0\%로 민간부문 노조조직률 11.2\%보다 높음.
        \item 공공부문노조의 높은 조직률과 사실상의 사용자인 정부를 상대로 한 단일한 대오 형성이 상대적으로 쉽다는 특성으로 인해 공공부문 노조의 영향력은 지속적으로 확대되고 있음. 
    \end{itemize}
\end{frame}

\begin{frame}[allowframebreaks]
    \frametitle{외국의 공공부문 고용관계}
    \begin{itemize}[<+->]
        \item 미국: 연방공공부문, 주 및 시의 공무원은 단체행동권을 제외하고 노동 2권 인정
        \begin{itemize}
            \item 연방공공부문 피고용인에게 단결권, 단체교섭권 부여. 단, 임금교섭권은 의회 의결이 필요하므로 제외. 극히 일부 주나 시를 제외하고 대부분 노동 2권을 허용
        \end{itemize}
        \item 독일: 공무원에게는 단결권만 부여. 비공무원인 사무직•노무직에게는 노동3권 부여
        \begin{itemize}
            \item 비공무원 사무직•노무직에게는 노동3권이 부여되나 파업 시 대체투입 허용
            \item 직원협의회를 설치•운영: 경영협의회와 유사하나 경제적 사항에 대한 불인정
        \end{itemize}
    \framebreak%
        \item 일본: 일반직 공무원에게 직원단체 결성 허용. 단, 단체협약체결권 및 단체행동권 배제
        \begin{itemize}
            \item 국영기업직원이나 지방공영기업 직원은 노조 조직 허용
            \item 인사원 설치•운영: 일반공무원의 급여, 기타 근무조건 개선 및 인사행정에 대한 개선 권고 권한 부여 $\implies$ 실제로 인사원의 권고안을 정부가 그래도 수용하여 실효를 거두고 있음
        \end{itemize}
    \end{itemize}
\end{frame}

\section{공무원 고용관계}
\begin{frame}[allowframebreaks]
    \frametitle{공무원의 노동기본권}
    \begin{itemize}[<+->]
        \item 과거에 체신, 국립의료원 기능직 등 사실상 노무에 종사하는 공무원만 노동 3권을 보장받음 (이외의 일반공무원은 노동3권 보장을 받지 못하였음)
        \item 1999년 6급이하 공무원들을 대상으로 노사협의만 할 수있는 공무원직장협의회가 허용됨
        \item 2006년 『공무원의 노동조합 설립 및 운영 등에 관한 법률』이 시행되면서 공무원노조 설립
    \end{itemize}
\framebreak%
    \begin{enumerate}
        \item 단결권: 단결권 보장. 다만, 업무성격상 군인 경찰 법관 등은 제외
        \item 교섭권과 협약체결권: 정부와 단체교섭하고 단체협약을 체결할 수 있는 권한 부여
        \item 단체행동권과 쟁의조정: 단체행동권을 부여 않음. 다만, 사실상 노무에 종사하는 기능직 공무원 등에는 부여
    \end{enumerate}
\end{frame}

\begin{frame}[allowframebreaks]
    \frametitle{공무원 노동조합 현황}
    \begin{itemize}[<+->]
        \item 과거 사실상 노무에 종사하는 공무원을 대상으로 한 전국우정노동조합, 국립중앙의료원노조 등 2개의 현업공무원 노조만 노동3권이 보장됨
        \item 2006년 공무원노조가 허용되면서 이후 공무원노동조합이 늘어나기 시작함
        \item 2021년 말 현재 공무원노조는 151개가 있으며 조합원 수는 351,830명이며 조직률은 74.1\%으로 민간부문보다 높음. 
    \end{itemize}
    \begin{table}
        \centering
        \resizebox{.65\textwidth}{!}{\relax
            \begin{tabular}{c|c|c|c}
\toprule
\textbf{구분} & \textbf{명칭 (약칭)} & \textbf{설립일} & \textbf{조합원수 (명)} \\
\midrule
\textbf{연합단체} & 대한민국공무원노동조합총연맹 (공노총) & ‘12.7.17 & 94,922 \\
 & 공무원노동조합연맹 (공무원연맹) & ‘12.10.18 & 67,130 \\
 & 전국시군구공무원노동조합연맹 (시군구연맹) & ‘13.3.25 & 36,589 \\
 & 전국시·도교육청공무원노동조합 (교육청노조) & ‘06.5.16 & 15,639 \\
 & 교육청노동조합연맹 (교육연맹) & ‘07.11.16 & 15,268 \\
 & 공무원노동조합전국연맹 (공노연) & ‘18.2.20 & 26,029 \\
 & 전국행정기관공무원노동조합연맹 (전국연맹) & ‘18.5.18 & 19,620 \\
\midrule
\textbf{전국단위} & 전국공무원노동조합 (전국노) & ‘18.3.26 & 150,667 \\
 & 전국통합공무원노동조합 (통합노조) & ‘15.6.16 & 23,470 \\
\midrule
\textbf{정부} & 국가공무원노동조합 (국공노) & ‘06.9.6 & 35,415 \\
\bottomrule
\end{tabular}

        }
        \caption{주요 공무원노조 현황}
    \end{table}
\end{frame}

\section{공공기관 고용관계}
\begin{frame}[allowframebreaks]
    \frametitle{공공기관 분류}
    \begin{itemize}[<+->]
        \item 공공기관: 공공서비스를 제공하지만 공무원과 달리 정부 부처 조직 밖에 있으며 정부로부터 일상적인 관리감독을 받지 않는 가운데 일정한 자율성을 갖고 정부정책을 수행하는 기관. 
        \item 공공기관은 정부가 법률상 사용자가 아니지만 사실상 사용자 역할을 함.  
    \end{itemize}
    \begin{table}
        \centering
        \resizebox{.3\textwidth}{!}{\relax
            \begin{tabular}{c|p{6cm}}
\toprule
\textbf{분류} & \textbf{세부분류} \\
\midrule
공기업 & 시장형 공기업 \newline 준시장형 공기업 \\
\midrule
준정부기관 & 기금관리형 준정부기관 \newline 위탁집행형 준정부기관 \\
\midrule
기타공공기관 & 공기업, 준정부기관이 아닌 공공기관 \\
\bottomrule
\end{tabular}

        }
        \caption{중앙정부 공공기관분류 기준}
    \end{table}
    \framebreak%
    \begin{itemize}[<+->]
        \item 지방공공기관: 중앙정부공공기관과 성격과 역할이 비슷하지만 지방자치단체가 설립함.
        \item 지방공공기관은 지방공기업과 지방자치단체 출자출연기관으로 구분할 수 있음.
    \end{itemize}
    \begin{table}
        \centering
        \resizebox{.3\textwidth}{!}{\relax
            \begin{tabular}{c|p{2cm}}
\toprule
\multicolumn{2}{c}{\textbf{구분}} \\
\midrule
\textbf{지방공기업} & 직영기업 \newline 지방공사 \newline  지방공단 \\
\midrule
\multicolumn{2}{c}{\textbf{지방출자출연기관}} \\
\bottomrule
\end{tabular}

        }
        \caption{지방자치단체 공공기관 분류}
    \end{table}
\end{frame}

\begin{frame}[allowframebreaks]
    \frametitle{공공기관 노동자의 노동기본권}
    \begin{enumerate}[<+->]
        \item 단결권과 단체교섭권: 단결권, 단체교섭권, 단체협약 체결권 등 보장
        \begin{itemize}
            \item 별도 제한없이 단결권, 단체교섭권, 단체협약 체결권 보장
            \item 정부 예산 영향을 받기 때문에 임금인상, 노동조건 개선에서 예산 제약이 있음
        \end{itemize}
        \item 단체행동권과 쟁의조정: 단체행동권을 행사할 수 있음. 다만, 공익사업이나 필수공익사업인 경우 쟁의행위 제약
    \end{enumerate}
\end{frame}

\begin{frame}[allowframebreaks]
    \frametitle{지방공기업 고용관계}
    \begin{itemize}[<+->]
        \item 최근 지방자치제가 성숙해지면서 지방자치단체수준 고용관계에 대한 관심이 높아지고 있음
        \begin{enumerate}[<+->]
            \item  전국의 고용관계는 중앙정부가 일괄적으로 관장하지만 지방의 고용관계는 현지의 상황을 반영하는 지방자치단체의 역할이 중요함.
            \item  지방자치단체가 해당 지역사정을 가장 잘 알고 있기 때문에 고용노동문제와 관련하여 세부적인 사항을 잘 반영할 수 있기 때문임.
            \item  해당지역 노동자는 많은 경우 해당 지방자치단체의 주민이기도 하다는 점에서 지방자치단체가 고용노동문제에 관심을 가져야 함은 당연함.
            \item 지방자치단체가 공공부문에서는 모범적인 사용자로서 합리적이고 모범적인 노사관계를 정립해 나갈 필요가 있음.
        \end{enumerate}
    \end{itemize}
\end{frame}

\begin{frame}[allowframebreaks]
    \frametitle{공공기관 고용관계 현황}
    \begin{itemize}[<+->]
        \item 노조: 대부분의 공공기관에 노조가 설립되어 있음. 
        \begin{itemize}
               \item 2020년 말 기준, 496개 공공기관 (공공기관의 운영에 관한 법률에 따른 중앙공공기관 및 『지방공기업법』에 따른 지방공기업 기준) 중 407개 기관에 노조가 설립되어 있으며, 조합원 수는 344,697명으로 노조 가입률은 65.7\%에 이름.
        \end{itemize}
        \item 한국노총과 민주노총 소속노조들은 함께 공공부문노동조합공동대책위원회 (공대위)를 구성하여 정부를 상대로 한 공동행동을 전개하고 있음. 
        \begin{itemize}
               \item 경쟁관계인 양대노총이 공동행동을 하는 이유는 공공부문의 특성상 사실상의 사용자가 정부라는 공통점이 있고 예산편성지침, 구조조정지침 등 정부지침에 따라 공공기관 노동조건이 크게 규정받는다는 점에 기인함.
        \end{itemize}
        \item 사용자: 공공기관은 민간기업과 달리 시장에 의한 통제를 안 받거나 덜 받기 때문에 시장 역할을 정부가 일부 대신하고 있음. 공공기관은 시장 기능 부재로 정부가 감독 기능을 대신 수행함.  기획재정부가 발표하는 예산편성지침은 사실상 공공기관 임금인상을 결정하는 역할을 하고 있음.
    \end{itemize}
\end{frame}

\section{교원 고용관계}
\begin{frame}[allowframebreaks]
    \frametitle{교원노동조합의 태동}
    \begin{itemize}[<+->]
        \item 교원 고용관계는 공공부문인 공립학교와 민간부문인 사립학교를 모두 포괄함. 교육의 공익성 때문에 공립학교와 사립학교 모두 공공부문에 준하는 고용관계를 유지하고 있음.
        \item 1960년 4•19혁명 이후 결성. $\implies$ 5•16군사 쿠데타 이후 해체. 
        \begin{itemize}
            \item 국•공립학교교원이나 사립학교교원 모두 노동3권을 인정받지 못하였음
            \item ∵ 『국가공무원법』제66조, 『사립학교법』제55조, 제58조의 적용
        \end{itemize}
        \item 1987년 법외조직으로 결성되었다가 1천 4백 여명 교직 박탈
    \framebreak%
        \item ILO, OECD 등 국제기구로부터 교원의 노동기본권 보장 권고 받음
        \item 1999년 『교원의 노동조합 설립 및 운영에 관한 법률』제정으로 합법적 조직으로 탄생
        \begin{itemize}
            \item 교원노동조합: 국·공·사립의 초·중등교원을 대상으로 임금·근로조건·후생복지 등 
        \end{itemize}
        \item 2022년 현재 교원노조는 전교조와 교사노동조합연맹을 양대축으로 하고 있음.
        \begin{itemize}
            \item 교수노동조합은 2020년 설립이 시작된 이후 아직까지 영향력 있는 큰 노동조합으로 성숙하지는 못한 상태임.
        \end{itemize}
    \end{itemize}
\end{frame}

\begin{frame}[allowframebreaks]
    \frametitle{교원 고용관계의 주요 특징}
    \begin{enumerate}[<+->]
        \item 노동3권 보장: 국$\cdot$공립, 사립 교원에게 노동2권 보장.
        \begin{itemize}
            \item  그러나 학생들의 학습권 보호를 위하여 파업. 태업 기타 업무의 정상적인 운영을 저해하는 일체의 쟁의행위를 불허하고 있어 단체행동권에는 제약이 따름.
        \end{itemize}
        \item 노동조합 설립: 특별시•광역시•도 단위 또는 전국단위에 한하여 설립. 전임자 인정
        \begin{itemize}
            \item 교육의 중립성을 고려하여 교원노조의 정치활동 금지
        \end{itemize}
    \framebreak%
        \item 교섭위원 구성
        \begin{itemize}
            \item 노동조합측 교섭위원: 노조 대표자와 그 조합원 
            \item 사용자측 교섭위원: 교육부장관, 시•도 교육감 또는 사립학교 설립•경영자
        \end{itemize}
        \item 교섭방식
        \begin{itemize}
            \item 통일교섭만 허용
            \item 교섭사항: 조합원의 임금•노동조건•후생복리 등 경제적•사회적 지위향상에 한정
        \end{itemize}
        \item 단체협약
        \begin{itemize}
            \item 교원노조와 사용자는 단체협약을 체결할 수 있음. 다만, 법령, 조례, 예산에 의해 규정되는 내용은 단체협약의 효력을 인정하지 아니하고 사용자 측의 성실이행 노력의무를 부여하고 있어 노조 단체교섭권에 대한 제약이 있음.
        \end{itemize}
    \end{enumerate}
    \begin{itemize}[<+->]
        \item 쟁의조정: 교원 노동쟁의를 조정·중재하기 위해 중앙노동위원회 내에 교원노동관계조정위원회를 두고 있음.
        \item 단체행동:  단체행동, 특히 파업은 학생들의 수업권 보장을 위해 제한하고 있음. 노조에서는 헌법상 집회시위의 자유, 의사표현의 자유를 내걸고 일부 제한적인 범위에서 단체행동을 하고 있어서 노동자인 교원의 노동기본권과 교육 서비스 수혜대상자인 학생과 학부모의 권리 상충을 둘러싼 논쟁은 앞으로도 지속될 것으로 보임. 
    \end{itemize}
\end{frame}

\begin{frame}[allowframebreaks]
    \frametitle{외국의 교원고용관계 사례}
    \begin{itemize}[<+->]
        \item 미국: 주, 시•군 정부마다 입법례가 다르나 대체로 단결권 인정, 일부 주 단체교섭권 인정 쟁의권에 대해서 대체로 인정하지 않음
        \item 독일: 공무원인 교원은 단결권만 인정하고 단체교섭권과 단체행동권은 부인 비공무원 교원에게는 노동3권 보장. 민간부문과 동일한 적용
        \item 일본: 지방공무원으로 단결권, 교섭권은 인정하지만 단체행동권은 불허
        \item $\implies$대체로 교원의 단결권과 단체교섭권을 인정하고 있으나 단체행동권은 일반적으로 금지하고 있음. 한국 교원노조법 입법사례는 외국 입법사례와 흡사함. 
    \end{itemize}
\end{frame}


%------------------------------------------------
\end{document}
%------------------------------------------------

