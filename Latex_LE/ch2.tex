%----------------------------------------------------------
%	PACKAGES AND THEMES
%----------------------------------------------------------
\documentclass[aspectratio=169,xcolor=dvipsnames,handout]{beamer}

\usetheme{Darmstadt}
\usecolortheme{seahorse}
\usepackage[hangul]{kotex}
\usepackage{hyperref}
\usepackage{graphicx, array, adjustbox}
\usepackage{booktabs, multicol, multirow}
\setbeamercovered{transparent}

%----------------------------------------------------------
%	TITLE PAGE
%----------------------------------------------------------

\title{한국의 고용관계}
\subtitle{노사관계의 이론과 실제}
\author{오성재}
\institute[CNU] 
{
    충남대학교 경제학과\\
}
\date{2024년 9월 9일} 

%----------------------------------------------------------
\begin{document}
%----------------------------------------------------------

\begin{frame}
    \titlepage
\end{frame}

\begin{frame}{목차}
    %\tableofcontents[hideallsubsections]
    \tableofcontents
\end{frame}

\begin{frame}{학습목표}
    \begin{itemize}[<+->]
        \item 우리나라 고용관계 발전과정과 각 단계별 특징 시대순으로 설명 할 수 있다.
    \end{itemize}
\end{frame}

\section{한국 고용관계의 발전과정}

\subsection{노동운동 태동기 (1876년--1919년)}

\begin{frame}{고용관계의 주요사건}
    \begin{itemize}[<+->]
            \item 1876년: 강화도 조약 체결, 조선 강제 개국
            \item 1888년: 함경도 초산금광에서 최초의 노동쟁의 발생
            \item 1898년: 함경도 성진에서 최초의 노동조합 결성
            \item 1910년: 한일합방
            \item 1919년: 3·1 독립운동 발생
    \end{itemize}
\end{frame}

\begin{frame}{특징 및 의미}
    \begin{itemize}[<+->]
        \item 우리나라는 일본자본주의의 침투에 의해 상품 및 원료시장의 식민지로 전락
        \item 임금노동의 형성은 각 개항장의 물동량 증대로 인한 부두노동자의 증가로부터 
        \begin{itemize}[<+->]
            \item 경우에 따라서 변두·접장·십장 등의 통솔 하에 무형의 조직으로 단결
        \end{itemize}
        \item 식민지에 필요한 가공업의 점차적 발달로 공장노동자의 형성
    \end{itemize}
\end{frame}

\subsection{일제하의 고용관계 (1920년--1945년)}

\begin{frame}{고용관계의 주요사건}
    \begin{itemize}[<+->]
        \item 1920년: 최초의 전국 노동단체인 조선노동공제회 결성
        \item 1922년: 사회주의 계열 조선노동연맹 결성
        \item 1928년: 영흥 총파업 (한국 최초의 지역 총파업)
        \item 1929년: 원산 노동자 총파업 (일제하 최대 규모)
        \item 1938년: 중일전쟁 발발, 노동조합 활동 금지
    \end{itemize}
\end{frame}
    
\begin{frame}{특징 및 의미}
    \begin{itemize}[<+->]
        \item 일제는 조선에서의 독점자본 확립 및 착취를 위해 다음의 정책을 시행 
        \begin{itemize}[<+->]
            \item 조선에서의 회사설립 허가주의 (조선회사령)를 철폐
            \item 적극적인 자본수출정책 추진
        \end{itemize}
        \item 노동자의 투쟁이라는 경제적 목적과 민족독립투쟁의 정치적 목적을 동시에 가지는 고용관계
        \begin{itemize}[<+->]
            \item 사회적·경제적 이익 수호 및 노동3권을 중심으로 노동자계급의 민주적 권리 확립  
            \item 노동운동의 국제적 연대강화를 통한 세계평화의 옹호 
        \end{itemize}
        \item 현실비판적·정치지향적·투쟁적 성격 
        \begin{itemize}[<+->]
            \item 일제의 탄압으로 지하운동화하여 공산주의와 연계
        \end{itemize}
    \end{itemize}
\end{frame}

% 3. 미군정하의 고용관계
\subsection{미군정하의 고용관계 (1945년--1948년)}

\begin{frame}{고용관계의 주요사건}
    \begin{itemize}[<+->]
        \item 1945년: 8·15 해방 및 미군정 개시
        \begin{itemize}[<+->]
            \item 미군정 노동쟁의 중재를 포함한 노동보호에 관한 법령 공포 
            \item 좌익계 조선노동조합전국평의회 (전평) 결성
        \end{itemize}
        \item 1946년: 최고근로시간과 연소노동에 관한 법령 공포 (주당 48시간), 대한노총 결성, 9월 총파업 생
        \item 1947년: 전평이 미군정에 의해 불법화, 보건사회부 산하 노동국 설립
    \end{itemize}
\end{frame}

\begin{frame}{특징 및 의미}
    \begin{itemize}[<+->]
        \item 노조의 활동이 자유롭게 활성화
        \item 정치적 소용돌이 속에서 노조간·노사간의 대립과 충돌 심화
        \begin{itemize}[<+->]
            \item 1945년 급진좌경 색채를 띤 조선노동조합전국평의회 (전평)이 결성
            \item 1946년 우익의 상층지도층은 대한노총을 결성 
        \end{itemize}
        \item 이 시기 노동운동은 지나치게 정치적이어서 노동자의 실질적인 생활수준과는 거리감 존재
    \end{itemize}
\end{frame}

% 4. 제1공화국의 고용관계
\subsection{제1공화국의 고용관계 (1948년--1960년)}

\begin{frame}{고용관계의 주요사건}
    \begin{itemize}[<+->]
        \item 1948년: 제1공화국 탄생
        \item 1950--1953년: 한국전쟁
        \item 1953년: 노동조합법, 노동쟁의조정법, 노동위원회법, 근로기준법 제정
        \item 1959년: 대한노총 분열에 이어 전국노동조합협회 (전노협) 결성
        \item 1960년: 4·19 혁명으로 제1공화국 종식, 대한노총과 전노협이 전노협으로 결합
    \end{itemize}
\end{frame}

\begin{frame}{특징 및 의미}
    \begin{itemize}[<+->]
        \item 미국의 신탁통치 영향으로 미국식 단체교섭의 고용관계 방식 채택 
        \item 노동자의 사회적 지위와 경제적 복지에 기여하지 못함
        \begin{itemize}[<+->]
            \item 전평을 타도한 대한노총은 친정부적 정치성향과 내부 갈등 존재
            \item 사회적·경제적·정치적 여건의 미성숙으로 노동운동 전개가 어려움
        \end{itemize}
        \item 물가에 상응하는 실질임금 보상, 체불임금 지급 등 임금문제가 주요 노동쟁의의 원인 
        \item 4·19혁명 이후 정치적·사회적 여건 미성숙으로 고용관계의 획기적 전환 불발
    \end{itemize}
\end{frame}

% 5. 경제개발기의 고용관계
\subsection{경제개발기의 고용관계 (1961년--1987년)}


\begin{frame}{고용관계의 주요사건}
    \begin{itemize}[<+->]
        \item 1961년: 5·16 쿠데타로 제3공화국 출범
        \begin{itemize}[<+->]
            \item 모든 노동조합은 해체후 산업별 노조로 전환, 한국노동조합총연맹 설립
        \end{itemize}
        \item 1963년: 보건사회부 노동국을 노동청으로 개편
        \item 1972년: 한국경영자총협회 설립
        \item 1975년: 근로기준법 개정 (5인 이상 사업에 적용/일부규정은 15인 이상에만 적용)
        \item 1980년: 5·17로 제4공화국 종식
        \begin{itemize}[<+->]
            \item 노동조합법, 노동쟁의조정법, 노동위원회 개정, 산별노조체제를 기업별노조로 전환
        \end{itemize}
        \item 1981년: 산업안정보건법 제정/노동청이 노동부로 승격
        \item 1986년: 최저임금법 제정
    \end{itemize}
\end{frame}

\begin{frame}{특징 및 의미}
    \begin{itemize}[<+->]
        \item 경제개발과 노동운동 (1961--1970): 5·16쿠데타 이후 정부의 고도경제성장추진 
        \begin{itemize}[<+->]
            \item 노동조합의 형태가 기업별 노조에서 산업별 노동조합 형태로 전환
            \item 노동행정의 효율화를 위해 주무기관을 노동국에서 노동청으로 승격
            \item 노동정책을 경제정책의 일환으로 인식, 노·사 모두 정부의 강력한 추진력에 압도 
        \end{itemize}
    \end{itemize}
\end{frame}

\begin{frame}{특징 및 의미 (계속)}
    \begin{itemize}[<+->]
        \item 노동기본권 제약하의 노동운동 (1971--1979): 수출 및 중화학공업의 육성, 국가안보 우선
        \begin{itemize}[<+->]
            \item 70년대 초반 비상사태 및 유신헌법 등을 근간으로 노동운동의 엄격한 규제, 
            \item 정부주도형의 고용관계 구축, 결과적으로 노동운동의 제약
            \item 근로조건의 개선 미흡, 산업간·학력간·남녀간의 임금격차 심화 등 근로자의 불만 표면화
            \item cf. 전태일 사건: 1970년 11월 동대문평화시장의 피복공장 재단사인 전태일이 근로기준법 준수를 요구하며 분신자살, 한국노동운동의 출발점
        \end{itemize}
    \end{itemize}
\end{frame}

\begin{frame}{특징 및 의미 (계속)}
    \begin{itemize}[<+->]
        \item 신군부정권의 억압과 노동운동 (1980--1983): 국가안정의 명목으로 노동운동을 과거보다 더 제약
        \begin{itemize}[<+->]
            \item 기업별 단위노조의 기능 강화, 노조 하부조직의 잦은 상부조직 비판
            \item 중소기업을 중심으로 첨예화
            \item 근로자들의 연대강화, 자연발생적 노동운동의 조직화, 재야노동운동세력 형성
        \end{itemize}
        \item 유화국면과 노동운동의 활성화 (1984--1987초): 정치적 유화기를 맞아 노동운동의 활성화 
        \begin{itemize}[<+->]
            \item 노동운동을 사회변혁의 중심으로 부각
            \item 노동운동의 조직과 투쟁노선을 둘러싼 논쟁과 실천활동 전개
        \end{itemize}
    \end{itemize}
\end{frame}

% 6. 민주화 이행기의 고용관계
\subsection{민주화 이행기의 고용관계 (1987년--1996년)}

\begin{frame}{고용관계의 주요사건}
    \begin{itemize}[<+->]
        \item 1987년: 6·29 민주화 선언과 노동자대투쟁
        \item 1990년: 전국노동조합협의회 결성
        \item 1991년: 국제노동기구 (ILO) 가입
        \item 1995년: 민주노총 결성
        \item 1996년: 대통령 직속 노사관계개혁위원회 출범, 노동법개정으로 대규모 시위 발생
    \end{itemize}
\end{frame}

\begin{frame}{특징 및 의미}
    \begin{itemize}[<+->]
        \item 노조 조직역량의 확대와 민주노조운동 기반구축 (1987--1988): 노동운동발전의 획기적 계기 
        \begin{itemize}[<+->]
            \item 최대 규모의 파업투쟁: 대중적 항쟁의 성격을 띄며 전국적·전산업적 범위에서 동시적으로 표출
            \item 임금·노동조건 개선을 비롯한 다양한 요구 분출, 기본권리 보장 강력 제기
            \item 자생적으로 파업투쟁이 발생하고 일부에서는 연대투쟁 시도
            \item 중화학공업, 생산직, 남성 노동자층이 주도세력으로 등장
            \item 87년 민주화투쟁을 계승하고 타 민중운동 및 사회운동발전의 기폭제 역할  
        \end{itemize}
    \end{itemize}
\end{frame}

\begin{frame}{특징 및 의미 (계속)}
    \begin{itemize}[<+->]
        \item 노동운동의 침체와 새로운 방향의 모색기 (1989--1995) 
        \begin{itemize}[<+->]
            \item  노동운동의 조건과 국가-자본-노동관계의 변화가 급속하게 이루어져 노동운동 방향전환 모색 
            \item  김영삼정권의 등장으로 노동운동에 대한 강압적 통제 완화
            \item  노동운동이 조직개편과 활동강화 등 재정비, 민주노동조합총연맹 (민주노총) 결성
        \end{itemize}
    \end{itemize}
\end{frame}

\begin{frame}{특징 및 의미 (계속)}
    \begin{itemize}[<+->]
        \item 노동법 개정과 총파업 투쟁 (1996--1997)
        \begin{itemize}[<+->]
            \item  1996년 말 노동법 개정에 대해 총파업 투쟁 결행: 97.3. 노동관계법 개정안 국회 통과
            \begin{itemize}
                \item 제3자개입금지조항 삭제
                \item 노조 정치활동금지조항 삭제
                \item 복수노조 단계적 허용    
                \item 노조 전임자 임금지급 금지
                \item 무노동·무임금 원칙 명문화 
                \item 정리해고제 법제화
                \item 변형근로기간제 도입
            \end{itemize}
            \item  집단적 노동기본권 강화 및 노동시장 유연화 등 제도적 기반 확보
        \end{itemize}
    \end{itemize}
\end{frame}

% 7. IMF 경제위기 이후 고용관계
\subsection{IMF 경제위기 이후 고용관계 (1997년-2007년)}


\begin{frame}{고용관계의 주요사건}
    \begin{itemize}[<+->]
        \item 1997년: 아시아외환위기와 IMF 구제금융
        \item 1998년: 김대중정부 출범, 노사정위원회 설치, 노동시장 유연화대책 시행 합의
        \item 1999년: 민주노총 합법화
        \item 2001년: 노사정위 (주 40시간 근로제 원칙적 합의)
        \item 2006년: 공무원노동조합 합법화
        \item 2007년: 비정규직 보호법 시행, 노사정위원회가 경제사회발전노사정위원회로 개편
    \end{itemize}
\end{frame}

\begin{frame}{특징 및 의미}
    \begin{itemize}[<+->]
        \item 1997년 외환부족사태를 IMF  및 국제자본의 금융지원으로 해결하고 금융기관의 부실화와 실물경제의 위기는 외자유치와 내국기업의 국제경쟁력 강화 및 국민의 고통분담 등으로 극복
        \begin{itemize}[<+->]
            \item 경제주체의 참여와 협력을 기반으로 하는 사회적 협의기구 (노사정위원회) 설치 운영
            \item 기업경쟁력 제고를 위해 신자유주의적 개혁을 확대 시행
        \end{itemize}
        \item 1998년 2월 ‘경제위기 극복을 위한 노사정 대타협’을 도출, 경제위기 극복에 큰 도움
        \item 2000년 이후 기업의 상시적 구조조정, 비정규직 양산 등으로 빈부격차 악화
        \begin{itemize}[<+->]
            \item 비정규직 등의 분규는 증가한 반면 대기업 분규는 줄어드는 노사분규의 양극화 초래
        \end{itemize}
    \end{itemize}
\end{frame}

\begin{frame}{특징 및 의미 (계속)}
    \begin{itemize}[<+->]
        \item 노무현정권은 공무원노조 설립 및 단체교섭권 인정, 비정규직법안 입법
        \begin{itemize}[<+->]
            \item 노동계는 공무원의 단체행동권 및 비정규직 법안 철폐 등을 요구
            \item 노사정의 갈등 격화
        \end{itemize}
        \item 노조조직률은 1989년 이후 지속적으로 감소
        \begin{itemize}[<+->]
            \item 일부 노조지도부의 부패, 노조의 강성 일변도의 극한적 투쟁 등으로 여론 악화
        \end{itemize}
        \item 2007년 비정규직 보호법이 시행됨으로써 비정규직의 처우 등에 대한 사회적 관심이 고조
    \end{itemize}
\end{frame}

\subsection{2008년-현재}

\begin{frame}{고용관계의 주요사건}
    \begin{itemize}[<+->]
        \item 2008년: 미국발 금융위기/이명박정부 출범
        \item 2009년: 「경제위기 극복을 위한 노사민정 합의문」 채택
        \item 2011년: 기업 내 복수노조허용/노조전임자 임금 지급금지 (타임오프제도) 시행
        \item 2013년: 박근혜 정부 출범
        \item 2015년: 9.15 노사정 대타협
        \item 2017년: 박근혜 대통령 탄핵/문재인 정부 출범
        \item 2018년: 경제사회발전노사정위원회를 경제사회노동위원회로 개편/주 52시간 근무시간제 실시
        \item 2020년: 코로나19 위기 극복을 위한 노사정협약 체결/사회적 거리두기 실시
        \item 2021년: 중대재해처벌법 시행
    \end{itemize}
\end{frame}

\begin{frame}{특징 및 의미}
    \begin{itemize}[<+->]
        \item 2008년 미국 발 부동산 위기로 촉발된 금융위기 발생, 노동시장 유연화 압력이 강화
        \item 2009년 미국 발 금융위기 극복을 위하여 『경제위기 극복을 위한 노사민정 합의문』을 채택함
        \item 2011년 기업 내 복수노조가 허용되고 노조 전임자의 임금지급금지제도 (타임오프제도)가 시행됨
    \end{itemize}
\end{frame}

\begin{frame}{특징 및 의미 (계속)}
    \begin{itemize}[<+->]
        \item 2017년 5월에 문재인정부 출범
        \begin{itemize}[<+->]
            \item 이전 정부가 만든 양대 지침 (저성과자 해고지침, 취업규칙완화 지침)의 폐기를 선언
        \end{itemize}
        \item 2018년 6월에는 사회적 대화 기능을 기존의 노사정을 넘어 청년, 비정규직, 여성 등 다양한 사회주체로까지 확대하고자 기존의 경제사회발전노사정위원회를 경제사회노동위원회로 개편
        \item 2018년 7월부터 주당 법정 근로시간을 이전 68시간에서 52시간으로 단축하는 내용을 골자로 하여 개정된 근로기준법이 종업원 300명 이상의 사업장과 공공기관을 대상으로 우선 시행
        \item 2020년 코로나19 위기 극복을 위한 노사정협약을 체결 (민주노총은 최종 불참함)
        \begin{itemize}[<+->]
            \item 사회적 거리두기 시행
        \end{itemize}
        \item 2021년 중대재해처벌법 시행
    \end{itemize}
\end{frame}

%------------------------------------------------
\end{document}
%------------------------------------------------
