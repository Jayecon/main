%----------------------------------------------------------
% PACKAGES AND THEMES
%----------------------------------------------------------
\documentclass[aspectratio=169,xcolor=dvipsnames,handout]{beamer}

\usetheme{Darmstadt}
\usecolortheme{seahorse}

\usepackage[hangul]{kotex}
\usepackage{hyperref}
\usepackage{graphicx, xcolor, amsmath}
\usepackage{booktabs, array, adjustbox, makecell, multicol, multirow}

\setbeamercovered{transparent}

%----------------------------------------------------------
% TITLE PAGE
%----------------------------------------------------------
\title{자료 가공하기}
\author{오성재}
\institute[CNU]
{\relax
    전북대학교 Stata 특강\
}
\date{2024년 11월 7일}

%----------------------------------------------------------
\begin{document}
%----------------------------------------------------------

\frame{\titlepage}

\begin{frame}
\frametitle{학습목표}
    \begin{itemize}[<+->]
    \item 추출한 rawdata를 가공하여 연구자에게 필요한 변수를 만드는 법을 익힌다.
    \end{itemize}
\end{frame}

\begin{frame}{목차}
    \small
    \tableofcontents[hideallsubsections]
\end{frame}

\section{do file}

\begin{frame}[fragile]
    \frametitle{do file}
    \begin{itemize}[<+->]
        \item 지금까지는 command 창에 하나의 명령어 만을 단순하게 입력했음.
        \item 그러나 자료 cleaning 및 회귀분석 같은 현실적인 사용에서는 여러 명령어가 연속적으로 사용됨.
        \item Stata는 여러 명령어를 do 확장자 파일로 저장함.
    \end{itemize}
    \begin{verbatim}
    do abc.do
    doedit abc.do
    \end{verbatim}
\end{frame}

\section{loop}

\begin{frame}[allowframebreaks, fragile]
    \begin{itemize}[<+->]
        \item macro는 특정 문구를 저장하는 이름.
        \item 다양한 macro 가운데 local, tempvar, tempfile, tempname 의 넷만 배운다.
        \item local: 숫자, 문자 (변수명, 파일명)등을 저
        \item tempvar: 임시변수명
        \item tempfile: 임시파일명
        \item tempname: 임시 매트릭스명
    \end{itemize}
    \frametitle{macro}
    \begin{verbatim}
    local x 1
    local y = 1
    local z1 `x' + 1
    local z2 "`y' + 1"
    local z3 `x' + `y' +1
    di "macro x = `x' and marco y = `y'" 
    di "macro x = " `x' " and marco y = "`y' 
    di "macro z1 = `z1', macro z2 = `z2' and macro z3 = `z3'" 
    di "macro z1 = "`z1', "macro z2 = "`z2' " and macro z3 = "`z3' 

    local x = foo
    local y bar
    di "macro x = `x' and marco y = `y'" 
    di "macro x = " `x' " and marco y = "`y' 
    di "macro z1 = `z1', macro z2 = `z2' and macro z3 = `z3'" 
    di "macro z1 = "`z1', "macro z2 = "`z2' " and macro z3 = "`z3' 

    local yr : disp %02.0f  = 7
    di "`yr'"
    \end{verbatim}
    \framebreak%
    \begin{verbatim}
    sysuse auto , clear
    tempfile tfile //임시파일
    preserve
        gen tag = 1
        save `tfile'
    restore
    append using `tfile' //파일간 수직결합
    de
    \end{verbatim}
\end{frame}


\begin{frame}[allowframebreaks, fragile]
    \frametitle{if loop}
    \begin{verbatim}
    di p
    di p[1]
    if p >= 3000 {
        sum
    }
    if p >= 6000 {
        sum
    }

    local year = 2007
    if "`year'" == "2007" {
        sum 
    }
    if "`year'" == "2006" {
        sum 
    }
    \end{verbatim}
\framebreak%
    \begin{verbatim}
    if "`year'" == "2006" {
        sum 
    }
    else{
       di "local year != 2006"
    }
    \end{verbatim}
\framebreak%
    \begin{verbatim}
    if "`year'" == "2006" {
        sum 
    }
    else if "`year'" < "2006" {
       di "local year < 2006"
    }
    else {
       di "local year > 2006"
    }
    \end{verbatim}
\end{frame}

\begin{frame}[allowframebreaks, fragile]
    \frametitle{foreach loop}
    \begin{verbatim}
    sysuse auto , clear
    foreach i of varlist p-f {
        di `i'
        di "`i'"
    }

    foreach i in foo bar{
        di "`i'"
    }

    local mylist foo bar
    foreach i of local mylist {
        di "`i'"
    }

    foreach i of local mylist {
        di `i'
    }
    \end{verbatim}
\end{frame}

\begin{frame}[fragile]
    \frametitle{forvalue loop}
    \begin{verbatim}
    forvalue i = 1/5 {
        di "`i'"
    }

    forvalue i = 1(2)9 {
        di "`i'"
    }
    \end{verbatim}
\end{frame}

\begin{frame}[allowframebreaks, fragile]
    \frametitle{Nested loop}
    \begin{verbatim}
    forvalue i = 7/9 {
        local yr : disp %02.0f  = `i'
        foreach j in foo bar {
            di "`j'`i'"
            di "`j'`yr'"
            di as text "Year == " as result "`yr'"
            }
    }
    \end{verbatim}
\framebreak%
    \small
    \begin{verbatim}
    forvalue i = 7/9 {
        foreach j in foo bar {
            if `i' == 7 {
                di "`j'`i'"
            }
            else {
                di "nope"
            }
        }
    }
    \end{verbatim}
\end{frame}

\section{data step1}%

\begin{frame}[allowframebreaks,fragile]
    \frametitle{some commands}
    \begin{itemize}[<+->]
        \item rename: 변수의 이름을 변경
        \item gen: 새로운 변수 만들기 (단순)
        \item label: 라벨 붙이기
            \begin{itemize}[<+->]
                \item label data 
                \item label var
                \item label define 
                \item label value
            \end{itemize}
    \framebreak%
        \item tostring: 숫자변수를 문자변수로
        \item destring: 문자변수를 숫자변수로 
        \item mvencode: 변수의 missing value를 특정한 값으로
        \item mvdecode: 변수의 특정한 값을 missing value로 
    \end{itemize}
 \end{frame}

\begin{frame}
    \frametitle{자료정리의 구조}
    \begin{enumerate}[<+->]
        \item 필요한 변수를 선정: local 사용 + tempfile, tempvar 등등
        \item 년도에 대한 forvalue루프 만들기
            \begin{itemize}[<+->]
                \item 년도별 rawdata 부르기: use
                \begin{itemize}[<+->]
                    \item 년도별로 필요한 변수가 상이할 수 있음: if문 + local 수정 + keep or drop
                    \item 년도별로 변수를 만들 경우도: gen + label
                \end{itemize}
                \item 그 밖의 예외처리
                \begin{itemize}[<+->]
                    \item 변수형 (숫자형, 문자형) 불일치: destrign, tostring
                    \item 변수별 missing 예외발생: mvdecode, mvencode
                \end{itemize}
            \end{itemize}
        \item  년도별 추출 및 수정된 파일들 합치기: append
    \end{enumerate}
\end{frame}

%----------------------------------------------------------
\end{document}
%----------------------------------------------------------
