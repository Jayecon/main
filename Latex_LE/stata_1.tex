%----------------------------------------------------------
% PACKAGES AND THEMES
%----------------------------------------------------------
\documentclass[aspectratio=169,xcolor=dvipsnames,handout]{beamer}

\usetheme{Darmstadt}
\usecolortheme{seahorse}

\usepackage[hangul]{kotex}
\usepackage{hyperref}
\usepackage{graphicx, xcolor}
\usepackage{booktabs, array, adjustbox, makecell, multicol, multirow}

\setbeamercovered{transparent}

%----------------------------------------------------------
% TITLE PAGE
%----------------------------------------------------------
\title{Stata를 활용한 자료분석 입문}
\subtitle{}
\author{오성재}
\institute[CNU]
{\relax
    전북대학교 Stata 특강\
}
\date{2024년 10월 17 일}

%----------------------------------------------------------
\begin{document}
%----------------------------------------------------------

\frame{\titlepage}

\begin{frame}
\frametitle{학습목표}
    \begin{itemize}[<+->]
    \item Stata를 처음 실행시키는 수준에서 출발하여 자료를 살펴보는 기초를 다진다.
    \end{itemize}
\end{frame}

\begin{frame}{목차}
    \small
    \tableofcontents[hideallsubsections]
\end{frame}

\section{살펴보기}

\begin{frame}[fragile]
\frametitle{폴더설정}
    \begin{verbatim}
    cmdlog using mylogfile, replace
    pwd

    creturn list
        di c(os)
        di c(sysdir_stata)

    cd ~
    \end{verbatim}
\end{frame}

\begin{frame}[fragile]
    \frametitle{자료 호출하기}
    \begin{verbatim}
    sysuse auto
        gen test = 1
        sysuse auto

    clear
    sysuse auto
        gen test = 1
    sysuse auto , clear
    \end{verbatim}
\end{frame}

\begin{frame}[fragile]
    \frametitle{자료 다루기의 기초}
    \begin{verbatim}
    list
    br
    
    de
    sum
    sum , detail
    sum , meanonly
    inspect
    \end{verbatim}
\end{frame}

\begin{frame}[allowframebreaks]
    \frametitle{변수의 종류}
    \begin{itemize}[<+->]
        \item 문자형: str\#
        \item 숫자형: byte, int, long, double
        \begin{itemize}[<+->]
            \item 실수형: 소득, 자산, 점수, 연령, 시점 등등.
            \item 범주형: 학력, 지역, dummy 등등.
        \end{itemize}
    \end{itemize}
\end{frame}

\begin{frame}[fragile]
    \frametitle{명령어 생략}
    \begin{verbatim}
    d
    de
    describe

    s
    su
    summarize
    \end{verbatim}
\end{frame}

\begin{frame}[fragile]
    \frametitle{조건부 명령 (in \& if)}
    \begin{verbatim}
    list
    list in 1/10
    list in 1/10 , sep(0)
    list price rep78 in 1/10
    list p r in 1/10
    list m in 1/10
    list p r in 1/10 if p < 5000
    list p r in f/5
    list p r in 40/l
    \end{verbatim}
\end{frame}

\section{연산자}

\begin{frame}[allowframebreaks]
    \frametitle{비교연산자}
    \begin{itemize}[<+->]
        \item grater than: $>$.
        \item less than: $<$.
        \item no more: $<=$.
        \item no less: $>=$.
        \item equal: $==$.
        \item not equal: $!=$.
    \end{itemize}
\end{frame}

\begin{frame}[allowframebreaks]
    \frametitle{논리연산자}
    \begin{itemize}[<+->]
        \item AND:\@ \&.
        \item OR:\@ $|$.
        \item NOT:\@! or \textasciitilde.
    \end{itemize}
\end{frame}

\begin{frame}[allowframebreaks, fragile]
    \frametitle{연산자 예시}
    \begin{verbatim}
    sum if price >= 4500
    sum if price < 4500
    sum if rep78 == 3
    sum if rep78 != 2
     
    sum if price >= 4500 & price <= 13000
    sum if inrange(price, 4500 , 13000)
    sum if price < 4500 | price > 13000
    sum if !inrange(price, 4500 , 13000)

    sum if rep78 == 3 and rep78 == 4
    sum if rep78 == 3 or rep78 == 4

    sum if rep78 == 3 & rep78 == 4
    sum if rep78 == 3 | rep78 == 4
    sum if inlist(rep78 , 3 ,4)
    \end{verbatim}
\end{frame}

\section{자세히 살펴보기}
\begin{frame}[allowframebreaks, fragile]
    \frametitle{이산변수 살펴보기}
    \begin{verbatim}
    tab rep78
    tab rep78 foreign
    tab foreign rep78
    tab turn rep78 foreign
    table turn rep78 foreign

    tab1 turn rep78 foreign
    tab2 turn rep78 foreign

    tab1
    tab1 _all

    tab f , sum(p)
    bysort f: sum p
    bys f: sum p

    tabstat p , by(r) stat(n mean sd min max)
    tabstat p , by(r) stat(n mean sd min max) missing
    \end{verbatim}
\end{frame}

\begin{frame}[allowframebreaks, fragile]
    \frametitle{연속변수 살펴보기}
    \begin{verbatim}
    hist price
    hist price , by(f)

    scatter rep78 weight
    scatter rep78 weight , by(f)

    kdensity price
    \end{verbatim}
\end{frame}


\begin{frame}[allowframebreaks, fragile]
    \frametitle{결측치}
    \begin{verbatim}
    tab foreign rep78
    tab foreign rep78 , mis

    sum if rep78 > 4
    sum if rep78 > 4 & rep78 < .
    sum if rep78 > 4 & !missing(rep78)
    \end{verbatim}
\end{frame}



\section{단순 조작}
\begin{frame}[allowframebreaks, fragile]
    \frametitle{자료의 순서 및 임시저장}
    \begin{itemize}[<+->]
        \item order: 변수의 순서변경
        \item sort: 관측치의 순서변경
        \item preserve: 자료의 임시저장
        \item restore: 임시저장된 자료로 복원
    \end{itemize}
    \framebreak\relax
    \begin{verbatim}
    preserve
    de
    order _all , alpha
    de
    order make foreign , first
    de

    restore , preserve
    list
    sort price
    list
    sort rep78 price
    list

    restore
    restore

    cmdlog close
    \end{verbatim}
\end{frame}


%------------------------------------------------
\end{document}
%------------------------------------------------

