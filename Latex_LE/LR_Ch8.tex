%----------------------------------------------------------
% PACKAGES AND THEMES
%----------------------------------------------------------
\documentclass[aspectratio=169,xcolor=dvipsnames,handout]{beamer}

\usetheme{Darmstadt}
\usecolortheme{seahorse}
\setbeamercovered{transparent}

\usepackage[hangul]{kotex}
\usepackage{hyperref}
\usepackage{graphicx, array, adjustbox, makecell}
\usepackage{booktabs, multicol, multirow, tabularx}

% font조정
%\usepackage{fontspec}
%\setmainfont{Times New Roman}
%\setmainhangulfont{NanumGothic}

% 문자열 대체{노사관계론 전용}
\usepackage{newunicodechar}
\newunicodechar{•}{$\cdot$}
\newunicodechar{➔}{$\implies$}
\newunicodechar{}{$\implies$}
\newunicodechar{∴}{$\therefore$}
\newunicodechar{∵}{$\because$}

%----------------------------------------------------------
% TITLE PAGㄱ
%----------------------------------------------------------
\title{무노조기업의 고용관계}
\subtitle{노사관계의 이론과 실제}
\author{오성재}
\institute[CNU]
{\relax
    충남대학교 경제학과\
    }
\date{2024년 11월 4일}

%----------------------------------------------------------
\begin{document}
%----------------------------------------------------------
\frame{\titlepage}

\begin{frame}{목차}
    \small
    \tableofcontents[hideallsubsections]
\end{frame}

\section{등장의 배경}
\begin{frame}
    \frametitle{무노조와 비노조}
    \begin{itemize}[<+->]
        \item 노조 영향력이 쇠퇴하면서 개별고용관계를 규율하는 법령과 제도의 중요성 증대
        \begin{itemize}
            \item ∵ 단체협약이 없는 상황에서 개별근로자를 보호하는 유력한 수단
        \end{itemize}
        \begin{itemize}
            \item 무노조 (non-union): 노조의 유무를 표현하는 가치중립적인 용어, 학문적 또는 공식적 용어로 적합. 유노조의 반대
            \item 비노조 (union-free): 사용자가 노조 없이 경영하도록 한 선택의 결과를 의미, 노조가 없는 상태를 선호하는 의미. 전략적으로 노조를 배제하는 정책을 펴는 기업에서 사용하는 용어
        \end{itemize}
    \end{itemize}
\end{frame}

\begin{frame}
    \frametitle{노조조직률의 하락}
    \begin{itemize}[<+->]
        \item 21세기 세계 고용관계의 가장 큰 이슈: 노동조합 조직률의 지속적 하락
        \begin{itemize}
            \item 지난 수십 년간 지속된 노조 쇠퇴가 앞으로도 지속   노조가 유명무실한 존재?
            \item 재 반등의 실마리  시장경제사회의 중심세력으로 존속? 
        \end{itemize}
        \item 전세계적인 노조 조직률 감소경향: 특히 상이한 문화와 정치제도, 서로 다른 발전단계, 상이한 노조형태 등을 감안할 때 노동운동의 침체는 노동계의 위기의식을 이끌어 내는 데 충분
        \item 노동쟁의도 감소 추세: 노조 조직률 하락과 맞물려 노조의 세력과 활동이 쇠퇴하는 것이 아닌가?
    \end{itemize}
\end{frame}

\begin{frame}[allowframebreaks]
    \frametitle{노조조직률 하락의 원인}
    \begin{itemize}[<+->]
        \item 노동조합 조직률의 하락현상은 주로 선진국에서 나타나는 공통적인 현상: 아래의 6가지 이유가 주 원인
    \end{itemize}
    \begin{enumerate}[<+->]
        \item  경제구조 변화
        \begin{itemize}
                \item 전통적으로 노조 조직률이 낮았던 서비스 산업과 화이트칼라 직종의 확대
                \item 노조 조직률이 낮은 여성노동력, 고령인력, 비정규직, 외국인인력 등의 증가 등 노동력 구성의 변화
        \end{itemize}
        \framebreak%
        \item  세계화와 경쟁의 격화: 자본의 이동은 자유롭고, 노동의 이동은 국가간 경계에 머무름
        \begin{itemize}
                \item 국경을 넘어 여러 국가에서 경영활동을 수행하는 다국적기업이 일반화
                \item 외국의 투자유인을 위해 다국적기업의 경영에 유리한 노동환경 조성 등
                    \item 2018년 이래로 미중간 무역갈등 심화와 2019년 코로나 판데믹 등의 영향으로 글로벌 밸류 체인이 붕괴되어 글로벌 양상이 글로컬하게 (Glocalization) 변모
                        \item $\implies$국가간 협업이 불안정해지면 리쇼어링 (reshoring) 현상이 강화됨에 따라 자국내 노조조직률 또는 노동환경에 영향을 미칠 가능성이 높아짐
        \end{itemize}
        \framebreak%
        \item  기술의 발달로 인한 새로운 고용형태의 증가: 비정규, 특고, 플랫폼노동의 증가
        \begin{itemize}
                \item 전통적으로 안정적인 피고용인들을 대상으로 조직되어왔던 노동조합이 기술진보로 확산되고 있는 프리랜서형 노동자의 등장으로 노조조직률의 위축에 영향을 미침 
        \end{itemize}
        \item  다양한 정체성 그룹의 등장
        \begin{itemize}
                \item 여성, 청년, 소수민족, 성소수자, 이민자, 노령인력 등 다양한 정체그룹 형성
                \item 노동과 자본의 대항적 관점에 기초하던 전통적인 노동조합의 정체성이 약화
        \end{itemize}
        \item  신자유주의와 노조에 대한 부정적인 여론
        \begin{itemize}
                \item 반노조적인 사회정치적 분위기를 배경으로 한 사용자들의 노조 회피전략 적극적 활용
                \item 노조 조직화 성향이낮은 지역이나 사업부문으로의 사업 중심을 이전
                \item 개별적인 인적자원관리 제도 활성화와 합리적인 고충처리절차 제공 등 노조 서비스 대체하는 제공
        \end{itemize}
        \framebreak%
        \item  개별노동법안의 노조대체현상
        \begin{itemize}
                \item 개별 노동기본권에 대한 정부의 입법 강화 추세로 집단 노사관계의 중요성이 약화: 고용평등법, 차별금지법, 모성보호법, 장애인고용촉진법, 성희롱 금지법, 직장갑질 금지법 등
                \item 부당한 대우에 대해 개별 노동자의 대응이 노조에서 기업을 상대로 소송하는 방식 증대 
        \end{itemize}
    \end{enumerate}
\end{frame}

\section{노조화}
\begin{frame}[allowframebreaks]
    \frametitle{노조화 이론}
    \begin{itemize}[<+->]
        \item 유노조기업은 무노조 기업의 직원들이 노조를 결성함으로써 발생 $\implies$ 노조화는 직무불만족과 노조의 수단성의 함수
        \[
            \text{노조화} = f(\text{직무불만족, 노조의 수단성})
        \]
        \item 직원들의 직무불만족이 높고 노조가 이를 해소할 수 있다고 믿을 때 노조화의 가능성이 높아짐
        \framebreak%
        \item 직무불만족: 고용안정, 임금, 상급자, 근로조건에 대한 불만 등 직무불만족이 높으면 직원들은 이직 (exit option)과 노조결성 (voice option)의 두 가지 선택에 직면하게 됨
        \item 노조의 수단성 (union instrumentality): 불만이 많은 직원들이 노조를 통하여 자신들의 직무 불만족을 해소할 수 있을 것이라고 믿음 $\implies$  노조의 수단성이 높으면 직무불만족한 직원은 이직보다는 노조결성을 택하게 됨
    \end{itemize}
\end{frame}

\begin{frame}[allowframebreaks]
    \frametitle{노조결성이유}
    \begin{itemize}[<+->]
        \item 공정하지 않고 일관성 없는 직원처벌로 인사문제를 경영층이 즉흥적으로 결정한다는 인상을 줌
        \item 경영층과 직원간의 공식, 비공식 의사소통 라인의 부재
        \item 중요 인사정책의 실시 시 직원의 의사를 전혀 반영치 않음
        \item 일부 그룹의 직원들을 편애, 우대하는 인사관행
        \item 회사의 경영성과에 대하여 직원들에게 전혀 알리지 않음
        \item 직원들의 불만이나 고충을 호소할 마땅한 채널이 존재하지 않음
        \item 직원의 정당한 불만이 전달되었음에도 여러 가지 이유로 해결되지 않음
        \item 직원들에 대한 훈련이 충분치 않아 직무수행에 애로 발생
        \item 능력위주의 승진원칙을 표명하고 객관적으로 능력이 있는 고참직원보다 그렇지 못한 신참직원을 승진시킴
        \item 사고나 질병의 위험이 높은 작업장 환경
        \item 제품 수요 감소에 따른 해고가능성을 줄일 노력을 기울이지 않음으로써 대량채용과 대량해고의 반복
        \item 경영층, 사무직, 생산직 지원간의 차별적인 대우: 식당, 주차장 등
        \item 임금과 수당이 경쟁기업보다 현저히 낮은 경우
    \end{itemize}
\end{frame}

\section{노조화 방지전략}
\begin{frame}
    \frametitle{개괄}
    \begin{itemize}[<+->]
        \item 무노조기업의 노조결성 반대 이유
        \begin{itemize}
            \item 경영권 침해 우려
            \item 노동비용 증가
        \end{itemize}
        \item 노조화 방지방안: 노조탄압, 노조회피, 노조대체 등
    \end{itemize}
\end{frame}

\begin{frame}[allowframebreaks]
    \frametitle{노조탄압}
    \begin{enumerate}[<+->]
        \item 노조파괴전문가 활용
        \begin{itemize}
        \item 노조탄압은 대부분 부당노동행위에 가까운 수단
            \item 노조파괴전문가 활용: 사용자가 신생노조의 결성 또는 기존 노동조합 활동을 적극 방해하려는 활동으로 노조파괴전문가를 활용
            \begin{itemize}
                \item 미국의 경우: 기존의 노동조합을 파괴하는 전문적 노조파괴전문 (union buster) 직업 존재
                \item 우리나라: 심각한 노사갈등을 겪고 있던 일부 기업에서 노조파괴전문가 고용
            \end{itemize}
            \item 이러한 활동을 적극적으로 하는 기업들은 전통적으로 무노조전략을 추구하는 기업들임
        \end{itemize}
        \framebreak%
        \item 사용자의 노조결성추진세력 해고
        \begin{itemize}
            \item 노조결성을 적극적으로 억압하고자 법망을 피하여 노조결성 추진세력 해고
            \item 우리나라의 경우: 사용자의 부당노동행위를 노동위원회가 판정하더라도 상습적이지 않으면 원상회복주의를 취하기 때문에 복직과 밀린 임금의 지급이 요구될 뿐임
            \item 노동운동으로 해고 후 복직된 근로자 입장에서는 보면 해고기간 중의 공백기간을 거친 후 다시 노조결성을 하여야 하므로 처음부터 다시 시작해야 하는 어려움이 있음
        \end{itemize}
        \framebreak%
        \item 교섭거부
        \begin{itemize}
            \item 정당한 사유없이 이미 결성된 노동조합과의 교섭 거 (부당노동행위에 속함) $\implies$ 적당한 이유를 들어 노조와의 교섭을 계속 피하고 무관심 속에서 노조를 대하면서 노조의 활동력을 저하시킴
            \item 단체교섭에 임하지만 교섭을 해태하는 전술을 쓰기도 함
        \end{itemize}
        \item 노조해산
        \begin{itemize}
            \item 노조 집행부를 반대하는 조합원을 모아서 노조해산 결의 유 (재적조합원의 과반수 찬성 시 가능)
            \item 사용자가 노조의 의사결정에 개입하였기 때문에 부당노동행위이지만 입증하기 곤란
        \end{itemize}
    \end{enumerate}
\end{frame}

\begin{frame}[allowframebreaks]
    \frametitle{노조회피}
    \begin{itemize}
            \item 노조회피: 노조 결성을 막고 최소한 확산되는 것을 막으려는 회사의 전략적 입장
    \end{itemize}
    \begin{enumerate}[<+->]
        \item  적극적 인적자원관리
        \begin{itemize}
            \item 경쟁대상인 노조기업보다 임금수준과 근로조건을 보다 더 좋게 제 (무노조 프리미엄)하여 노조 결성 의지를 약화. 즉, 근로자의 직무만족도를 향상시켜 노조를 원하지 않도록 하는 효과
        \end{itemize}
        \framebreak%
        \item  병렬형 관리 (double breasting)
        \begin{itemize}
            \item 유노조 사업장과 무노조 사업장을 함께 가진 기업에서 노조의 영향을 약화시키기 위해 유노조 사업장은 축소하고 무노조 사업장은 확대하여 관리 $\implies$ 중장기적으로 무노조 사업장에 투자와 고용을 늘리고 반대로 유노조 사업장은 투자 감축 및 고용 축소로 자연스럽게 노조 위축을 유도
        \item 또는 노조가 있는 공장을 국내외의 무노조지역으로 이전하여 노조를 제거하는 방법도 있음
        \end{itemize}
    \end{enumerate}
\end{frame}

\begin{frame}
    \frametitle{노조대체 (union substitution)}
    \begin{itemize}[<+->]
        \item 노조의 순기능을 대신할 수 있는 대안적 의사소통기 (alternative voice channel)을 운영하여 노조 결성을 막으려는 방법
        \begin{itemize}
            \item 노와 사가 공동체임을 강조하고 조직에 대한 충성심을 고취시킴
            \item 근로자의 고충처리나 제안을 적극 활용
            \item $\implies$ 노사협의회와 같은 조직을 적극 활용
        \end{itemize}
        \item 근로자의 의견을 반영하는 제 (예: 평사원협의회, 청년이사회, 청년중역회 등)를 통해 노조의 필요성을 약화시키고 노조결성 요구를 낮추기도 함
    \end{itemize}
\end{frame}

\begin{frame}
    \frametitle{무노조기업의 유형과 특징}
    \begin{table}
        \centering
        \resizebox{.8\textwidth}{!}{\relax
            \begin{tabular}{p{3cm} p{6cm} p{9cm}}
        \toprule
        \textbf{개념} & \textbf{개념} & \textbf{특징} \\
        \midrule
        \textbf{철학적 무노조}
            & CEO의 인재경영에 대한 철학을 갖고 있으며 무노조 경영은 목표가 아닌 경영철학의 부산물 
            & - 우수한 인적자원관리제도가 노조의 존재를 대체 \newline - 무노조프리미엄 \\
            \midrule
        \textbf{정책적 무노조}
            & 무노조경영을 인적자원관리의 목표로 천명하고 노조회피전략을 명시적으로 수립 실행
            & - 직원들을 잘 대우해줘 직무만족도를 향상시켜 노조발생을 억제하는 고진로 (high road)식 방식과 노조결정을 사전에 간파하여 노조결정움직임을 수단과 방법을 가리지 않고 탄압하는 저진로 (low road)식 방식이 있음 \newline - 노조의 대안으로 노사협의회 등과 같은 무노조직원대표조직을 적극 활용 \\
            \midrule
        \textbf{종교적 무노조}
            & 종교적인 믿음을 바탕으로 노사가 합심하여 기업을 운영 
            & 경영자는 노조가 경영자와 직원간의 종교적 화합을 저해하는 불필요한 제삼자로 인식하므로 노조의 결성을 명시적 묵시적으로 금지 \\
            \midrule
        \textbf{영세 무노조}
            & 노조도 없으며 우수한 인적자원관리제도도 없는 기업군 
            & 경영에 대한 전문지식이 부족한 대부분의 중소영세기업에서 나타남. 블랙홀 또는 블리크 하우스라고도 함 \\
            \bottomrule
    \end{tabular}

        }
    \end{table}
\end{frame}

\section{노사협의회}

\begin{frame}[allowframebreaks]
    \frametitle{노사협의회의 성격}
    \begin{itemize}[<+->]
        \item 작업장 단위에서 사용자와 근로자가 작업에서의 문제해결과 공동관심사를 협의하는 제도
        \begin{itemize}
            \item 근로자들의 의견이 경영에 반영된다는 점에서 종업원참여제도의 일종
            \item 단체교섭이 노사간 이익대립 경향인 반면, 노사협의회는 작업장 단위의 종업원 참여를 목표로 설립
        \end{itemize}
        \begin{exampleblock}{국가별 노사협의회의 형태}
            \begin{itemize}
                \item 노사협의회 (labor-management committee): \linebreak 한국, 프랑스, 벨기에 등. 노사대표로 구성
                \item 작업장평의회 (works council): \linebreak 독일, 오스트리아, 네덜란드, 스페인 등. 근로자 대표만으로 구성
            \end{itemize}
        \end{exampleblock}
        \item 일부 유럽과 아시아 등 법률로 강제, 영국과 미국은 노사간의 합의로 실시
        \item 한국의 노사협의회 (『근로자참여 및 협력증진에 관한 법률』) 
        \begin{itemize}
            \item 법률로 시행이 강제
            \item 근로자와 사용자가 참여와 협력을 통하여 근로자의 복지증진과 기업의 건전한 발전을 도모함을 목적으로 구성하는 협의기구
        \end{itemize}
    \end{itemize}
\end{frame}

\begin{frame}
    \frametitle{단체교섭과의 비교}
    \begin{table}
        \centering
        \resizebox{.8\textwidth}{!}{\relax
            \begin{tabular}{p{2cm} p{7cm} p{7cm}}
\toprule
 & \textbf{노사협의회} & \textbf{단체교섭} \\
\midrule
\textbf{목적} & 노사공동의 이익증진과 산업평화 도모 & 근로조건의 유지와 개선 \\ \midrule
\textbf{배경} & 노동조합의 설립 여부와 관계없이 쟁의행위라는 위협의 배경없이 진행 & 노동조합 및 기타 노동단체의 존립을 전제로 하고 자력구제로서의 쟁의의 배경 \\ \midrule
\textbf{당사자} & 근로자 대표와 사용자 & 노동조합의 대표자와 사용자 \\ \midrule
\textbf{대상사항} & 기업의 경영이나 생산성 향상 등과 같이 노사간 이해가 공통 & 임금 근로시간 기타 근로조건에 관한 사항처럼 이해가 대립 \\ \midrule
\textbf{결과} & 법적 구속력 있는 계약체결이 이루어지지 않을 수 있음 & 단체교섭이 원만히 이루어진 경우 단체협약 체결 \\
\bottomrule
\end{tabular}

        }
        \caption{노사협의회와 단체교섭의 비교}
    \end{table}
\end{frame}

\begin{frame}
    \frametitle{단체교섭과의 관계}
    \begin{itemize}[<+->]
        \item 노사협의회와 단체교섭의 관계: 노동조합의 단체교섭이나 그 밖의 모든 활동은 노사협의회법에 의하여 영향을 받지 않음. $\because$ 근로자참여 및 협력 증진에 관한 법률 (근참법) 제5조 (노동조합과의 관계).
        \begin{itemize}
            \item 분리형: 단체교섭과 노사협의회를 별도의 제도로 분리하여 운영하는 방식
            \begin{itemize}
                \item 노사협의회에서는 단체교섭사항을 다루지 않는 방식
            \end{itemize}
            \item 연결형: 단체교섭과 노사협의회 제도를 유기적으로 운영하는 방식
            \begin{itemize}
                \item 단체교섭과 노사협의회를 각각 별도의 제도로 분리해 운영하지만
                \item 단체교섭사항에 대해 노사협의회에서 예비적으로 의견교환과 절충을 행함
            \end{itemize}
            \item 대체형: 두 제도간 구분없이 운영하는 방식
            \begin{itemize}
                \item 단체교섭과 노사협의회 양 제도를 서로 구분하지 않고 노사협의회에서 단체교섭사항까지 논의하는 방식
            \end{itemize}
        \end{itemize}
    \end{itemize}
\end{frame}

\begin{frame}
    \frametitle{구성}
    \begin{itemize}[<+->]
        \item 『근로자참여 및 협력증진에 관한 법률』에 의하면 30인 이상의 사업장에서 반드시 구성해야 함
        \begin{itemize}
            \item 노사협의회 구성
            \item 근로자와 사용자를 대표하는 동수의 위원으로 구성하되 그 수는 각 3--10명 이내
            \item 의장은 위원 중에서 호선
            \item 근로자위원
            \begin{itemize}
                \item 근로자의 과반수로 조직된 노동조합이 있는 경우에는 노조 대표자와 그 노동조합이 위촉
                \item 노동조합원이 과반수를 넘지 못하거나 노동조합이 없는 경우에는 근로자의 과반수가 참여하여 직접·비밀·무기명투표로 선출
            \end{itemize}
            \item 위원의 임기: 3년으로 하되 연임이 가능
            \item 비상임·무보수
        \end{itemize}
    \end{itemize}
\end{frame}

\begin{frame}
    \frametitle{운영 및 임무}
    \begin{enumerate}[<+->]
        \item 운영
        \begin{itemize}
            \item 3개월마다 정기적으로 개최
            \item 과반수 이상의 출석으로 개최하고 출석위원 3분의 2 이상의 찬성으로 의결
        \end{itemize}
        \item 임무
        \begin{itemize}
            \item 보고사항: 경영정보 공유의  성질을 가진 사항
            \item 협의사항: 노사가 상호 협의하여 합의에 도달할 수 있는 생산·노무·인사관리에 관한 사항
            \item 의결사항: 반드시 협의회의 의결을 거쳐야만 시행할 수 있는 사항
            \begin{itemize}
                \item 교육훈련 및 능력개발 기본계획 수립, 복지시설 설치 및 관리, 사내근로복지기금 설치, 각종 노사 공동기구의 설치 관리 등
            \end{itemize}
        \end{itemize}
    \end{enumerate}
\end{frame}

\begin{frame}
    \frametitle{고충처리제도}
    \begin{itemize}[<+->]
        \item 고충처리위원: 상시 30인 이상의 근로자를 사용하는 사업장
        \item 구성: 노사를 대표하여 3인 이내의 위원으로 구성
        \begin{itemize}
            \item 노사협의회가 설치되어 있는 경우: 협의회가 그 위원 중에서 선임
            \item 노사협의회가 없는 경우: 사용자가 위촉
        \end{itemize}
        \item 고충 처리: 근로자로부터 고충사항을 청취한 때에는 10일 이내에 조치사항 기타 처리결과를 당해 근로자에게 통보하여야 함
        \begin{itemize}
            \item 고충처리가 곤란한 경우에는 협의회에 부의하여 협의·처리
        \end{itemize}
    \end{itemize}
\end{frame}


\section{노동시장의 새로운 행위자}
\begin{frame}
    \frametitle{등장배경}
    \begin{itemize}[<+->]
        \item 노동이해대변의 새로운 행위자 등장배경
        \begin{itemize}
            \item 지속적인 노동조합 조직률의 하락
            \item 사회양극화로 인한 주변부 및 비정규직 근로자의 증가
            \item 인터넷, SNS 등 새로운 압력 제기 방식의 등장 (Cyber Unionism)
        \end{itemize}
        \item 다양한 명칭으로 불림
        \begin{itemize}
            \item 시민사회단체 (Civil Society Organizations) 준노조 (Quasi-Unions), 노동인권단체 (Labor Rights Organizations), 노동NGOs (Labor NGOs) 등등
            \item 알바노조, 청년유니온, 청소년유니온, 노년유니온, 여성민우회, 외국인근로자쉼터 등등 
        \end{itemize}
    \end{itemize}
\end{frame}

\begin{frame}[allowframebreaks]
    \frametitle{준노조의 특징}
    \begin{block}{준노조}
        현대 경제구조의 다양한 고용관계 하에서 기존의 방식으로 대표되지 못하는 집단의 이해관계를 대변하기 위해, 일시적이며 다양한 조건 하에서 서로 연대하는 작은 조직, 회원제 조합 및 현상 그 자체
    \end{block}
    \begin{itemize}[<+->]
        \item 준노조의 구분
        \begin{itemize}
            \item 회원조직
            \item 후원조직
            \item 회원조직과 후원조직을 겸하는 조직
        \end{itemize}
        \framebreak%
        \item 준노조의 특징
        \begin{itemize}
            \item 기존 노조의 관심이 부족한 영역에 초점을 맞춤
            \item 거리에서의 목격 (witnesses on the street)
        \end{itemize}
        \item 준노조의 활동
        \begin{itemize}
            \item 주장 지향 활동: 정치적 압력 행사, 여론조성, 정치적 투쟁
            \item 서비스 지향 활동: 상호부조, 노동교육 및 법률 자문
        \end{itemize}
    \end{itemize}
\end{frame}

\begin{frame}
    \frametitle{노동조합과의 비교}
    \begin{table}
        \centering
        \resizebox{.8\textwidth}{!}{\relax
            \begin{tabular}{p{3cm} p{6cm} p{9cm}}
        \toprule
        \textbf{구분} & \textbf{노동조합} & \textbf{준노조} \\
        \midrule
        \textbf{지지층} & 
        경제적인 정체성(계층, 직업, 산업) \newline 핵심 노동자에 집중 &
        정치적·사회적인 정체성 \newline 주변부 노동자에 집중 \\
        \midrule
        \textbf{이해관계 대변} & 
        공통 규칙을 통해 추구하는 집단의 이익, 작업장·기업 수준의 이해관계 &
        유연한 수단을 통해 추구하는 개인과 집단의 다양한 이익, 노동시장 수준의 이해관계 \\
        \midrule
        \textbf{노동자와의 관계} & 
        공식적으로 조직된 노동자 조직 \newline 작업장·기업 수준의 조직화 &
        비조직화된 노동자 \newline 노동시장 수준의 조직화 \\
        \midrule
        \textbf{사용자와의 관계} & 
        단체교섭 활용 & 
        법과 제도 활용 \\
        \midrule
        \textbf{정부와의 관계} & 
        정당정치 활용, \newline 집단 노동관계법 개혁에 중점 &
        정치적 압력 행사, \newline 개별 근로관계법 개혁에 초점 \\
        \bottomrule
    \end{tabular}

        }
        \caption{노동조합과 준노조의 비교}
    \end{table}
\end{frame}

\begin{frame}
    \frametitle{한계와 성패요인}
    \begin{itemize}[<+->]
        \item 준노조의 한계
        \begin{itemize}
            \item 활동 기간이 단기에 그치는 경향
            \item 재정적으로 취약
            \item 영향력이 미약
            \item 본래 활동 목적에서 이탈하거나 목적이 변질될 가능성
            \item 소규모 지역단위 위주의 활동
        \end{itemize}
        \item 준노조의 성패 요인
        \begin{itemize}
            \item 이해대변 집단의 수요를 효과적으로 반영할 것 
            \item 계속 참여자나 지원자를 확대하여 재정적으로 안정된 지속가능한 조직으로 유지
            \item 사회적으로 여론의 관심을 집중
        \end{itemize}
    \end{itemize}
\end{frame}

\section{개별적 근로관계에 대한 노동법 이슈}

\begin{frame}[allowframebreaks]
    \frametitle{근로기준법상의 보호}
    \begin{itemize}[<+->]
        \item 경영상의 이유로 해고가 가능하기 위한 3가지 조건을 준수하고 최소한 30일 전에 해고 예고
        \begin{itemize}
            \item 긴박한 경영상의 필요가 인정되어야 하며
            \item 사용자는 해고를 피하기 위한 노력을 다하여야 하며
            \item 합리적이고 공정한 해고의 기준을 정하고 이에 따라 그 대상자를 선정
        \end{itemize}
        \item 탄력적 근로시간제와 선택적 근로시간제
        \begin{itemize}
            \item 탄력적 근로시간제: 2주 혹은 3개월 단 (6개월짜리도 최근 신설)로, 주당 52시간·일 12시간을 넘을 수 없음
            \item 선택적 근로시간제: 1달 이내 (연구개발 3개월짜리 최근 신설), 근로자가 근무시간 선택·결정
        \end{itemize}
        \item `여성'과 15세 미만의 `소년'에 대한 보호 규정
        \begin{itemize}
            \item 15세 미만의 소년은 원칙적으로 근로자로 사용하지 못함 
            \item 18세 미만인 경우에도 친권자나 후견인의 동의 필요
            \item 18세 미만자와 여성의 경우에는 임산부는 도덕상 또는 보건상 유해·위험한 사업에 사용하지 못함
        \end{itemize}
    \end{itemize}
\end{frame}

\begin{frame}
    \frametitle{파견근로자 보호}
    \begin{itemize}[<+->]
        \item 「 파견근로자 보호 등에 관한 법률」
        \begin{itemize}
            \item 파견 대상이 될 수 없는 업무: 제조업의 직접생산공정 업무와 건설공사현장 업무, 하역업무, 선원의 업무, 유해하거나 위험한 업무 등
            \item 파견기간은 원칙적으로 1년을 초과할 수 없으나 당사자 합의시 1년 연장가능 (1+1 시스템)
            \item 무분별한 파견근로 활용 방지를 위해 일정한 조건에 해당할 경우 사용사업주가 파견근로자를 직접 고용하도록 규정
        \end{itemize}
        \item 「기간제 및 단시간근로자 보호 등에 관한 법률」
        \begin{itemize}
            \item 기간제 근로자도 총 고용기간을 2년으로 한정하고 2년이 되면 사용자는 기간제 근로자와의 계약을 종료하거나 `기간의 정함이 없는 근로계약' (무기계약)을 체결한 근로자로 고용하여야 함
        \end{itemize}
    \end{itemize}
\end{frame}

\begin{frame}
    \frametitle{남여 고용평등}
    \begin{block}{『남여고용평등과 일·가정 양립지원에 관한 법률 (남녀고용평등법)}
        고용에서 남녀평등 기회와 대우를 보장하고 모성보호와 여성 고용을 촉진하여 남녀 고용 평등을 실현하고 근로자의 일과 가정의 양립을 지원하기 위해 제정된 법
    \end{block}
    \begin{itemize}
        \item 근로자를 모집하거나 채용시에 남녀 차별을 금지
        \item 동일한 사업내의 동일 가치 노동에 대해서는 동일한 임금을 지급
        \item 복리후생이나 교육․배치 및 승진, 정년․퇴직 및 해고 등에 있어서도 남녀 차별 금지
        \item 직장내 성희롱 금지, 직업능력개발과 경력단절여성에 대한 고용 촉진 의무
        \item 모성보호: 배우자 출산휴가 및 난임치료휴가, 육아기 근로시간 단축 등 
    \end{itemize}
\end{frame}

\begin{frame}[allowframebreaks]
    \frametitle{장애인고용촉진}
    \begin{itemize}[<+->]
        \item 장애인고용촉진등에 관한 법률
        \begin{itemize}
            \item 장애인의 고용촉진과 직업재활을 위해 국가와 지방자치단체의 책임을 부과
            \item 직장내에서 장애인에 대한 인식개선을 위한 교육실시 의무
            \item 장애인에 대해 상시 근로자 50인 이상이 되는 기업에 의무고용률을 부과
            \item 장애를 이유로 한 차별을 금지하고 장애를 이유로 차별받은 사람의 권익보호를 강화
            \item 장애인 의무고용률 2022년 현재 민간기업 3.1%, 공무원/공공기관 3.6%
        \end{itemize}
        \framebreak%
        \item 『장애인 차별금지 및 권리구제 등에 관한 법률』
        \begin{itemize}
            \item 사용자가 모집이나 채용, 임금 및 복리후생, 교육배치승진전보, 정년퇴직해고 등에 있어서 장애인을 차별하지 않도록 규정 
            \item 장애인이 직무를 수행함에 있어서 장애인이 아닌 사람과 동등한 노동조건에서 일할 수 있도록 필요한 정당한 편의를 제공할 의무를 사용자에게 부과
        \end{itemize}
    \end{itemize}
\end{frame}

\begin{frame}[allowframebreaks]
    \frametitle{외국인 근로자 보호}
    \begin{itemize}[<+->]
        \item 1993년 외국인 산업연수제도 도입
        \begin{itemize}
            \item 산업연수생제도 하에서 외국인 노동자는 연수생이라는 신분을 이유로 `노동자성'을 인정받지 못함
            \item 근로기준법상 보호를 받지 못하여 저임금과 높은 노동강도, 노동권과 폭력 등 인권침해 등으로 사업장을 이탈 불법 체류자가 되는 경우 증가
        \end{itemize}
        \framebreak%
        \item 2003년 외국인근로자의 고용허가제 도입
        \begin{itemize}
            \item 취업기간 동안에는 내국인과 동일하게 건강보험과 산재보험 등 4대 보험의 적용
            \item 부당한 차별금지
            \item 외국인 노동자의 취업활동은 입국한 날부터 3년, 1회에 한해 2년이내에서 연장 가능 (3 + 2 시스템) 
            \item 현 외국인 노동자 고용허가제의 문제점: 사업장 변경 제한에 관한 사항, 체류기간 만료 후에 불법체류 상태에서의 노동
        \end{itemize}
    \end{itemize}
\end{frame}

\begin{frame}
    \frametitle{산업안전과 산업재해}
    \begin{itemize}[<+->]
        \item 산재사고 사망률이 경제협력개발기구 (OECD) 회원국 중 가장 높은 편
        \begin{itemize}
                \item 산업재해가 중소기업과 하청업체 중심으로 발생: `위험의 외주화'
        \end{itemize}
        \item 사업장의 안전과 관련된 2가지 법적 규율 방안
        \begin{itemize}
            \item 「산업안전보건법」: 산업재해를 예방하고 쾌적한 작업환경을 조성하여 노동자의 안전 보건을 유지·증진
            \item 「산업재해보상보험법」: 노동자가 업무상 재해를 입은 경우 신속하고 공정한 보상
        \end{itemize}
    \end{itemize}
\end{frame}

%------------------------------------------------
\end{document}
%------------------------------------------------
