%----------------------------------------------------------------------------------------
%	PACKAGES AND THEMES
%----------------------------------------------------------------------------------------
\documentclass[aspectratio=169,xcolor=dvipsnames,handout]{beamer}
\usetheme{Darmstadt}
\usecolortheme{seahorse}

\usepackage[hangul]{kotex}

\usepackage{hyperref}
\usepackage{amsfonts, amssymb}
\usepackage{graphicx} % Allows including images
\usepackage{array, booktabs, multicol, multirow} % Allows the use of \toprule, \midrule and \bottomrule in tables
\setbeamercovered{transparent}

\newcommand{\R}{\mathbb{R}}
\newcommand{\y}{\mathbf{y}}

%----------------------------------------------------------------------------------------
%	TITLE PAGE
%----------------------------------------------------------------------------------------

\title[자유지상주의]{자유지상주의} % The short title appears at the bottom of every slide, the full title is only on the title page
\subtitle{경제정의와 불평등}

\author[오성재]{오성재}

\institute[HNU] % Your institution as it will appear on the bottom of every slide, maybe shorthand to save space
{
    한남대학교 \\
    탈메이지 교양학부 \\
}
\date{\today} % Date, can be changed to a custom date


%----------------------------------------------------------------------------------------
%	PRESENTATION SLIDES
%----------------------------------------------------------------------------------------

\begin{document}

\begin{frame}
    \titlepage
\end{frame}

\begin{frame}{목차}
    \tableofcontents
\end{frame}

\section{들어가늘 말}

\begin{frame}[<+->]
\frametitle{자유지상주의}
    \begin{itemize}
        \item 자유지상주의는 개인의 삶에 국가에 의한 법에 강압적 지배가 없어야 한다는 사상 중 하나.
        \begin{itemize}
            \item 모든 개인은 자신의 행동이 평등한 타인의 자유권을 침해하지 않는 한, 스스로 무엇을 할 것인지 결정할 자유가 있음.
        \end{itemize}
    \item 국가는 구성원들의 활동의 자유가 서로 일관되게 하기 위해서만 간섭.
        \begin{itemize}
        \item 국가는 폭력, 절도, 기만 등으로부터 구성원을 보호하는 역할.
        \item 가부장적 정부(예: 안전벨트 착용 의무화, 마약 사용 금지, 결혼에 대한 특정 견해 시행 등)의 역할을 하거나 재분배를 추구해서는 안됨.
        \item 그러한 국가권력 행사는 구성원 일부의 자유를 부당하게 제한하기 때문.
        \end{itemize}
    \item 이러한 자유지상주의 원칙은 단순하면서도 강력함.
    \end{itemize}
\end{frame}

\begin{frame}[<+->]
\frametitle{자유지상주의의 권한}
    \begin{itemize}
        \item 자유지상주의 원칙은 권한의 한계에 대한 원칙.
        \item 권한의 원초적 한계는 개인과 국가에 동일하게 적용.
        \item 국가는 권한 자체의 제한이나 국가에 추가 권한이 부여된 계약에서 비롯된 권한 외에는 어떠한 권한도 없음.
        \begin{itemize}
            \item 국가는 모든 개인이 스스로의 권리를 존중받는 데 이용될 수 있음.
            \item 국가는 사회구성원과의 계약에 의해 추가적인 권한을 가질 수 있음.
            \item 그러나 국가는 그러한 계약이 없으면 사회구성원의 일관된 자유 행사를 보장하는 것 이상의 권한을 가질 수 없음.
        \end{itemize}
    \end{itemize}
\end{frame}

\begin{frame}[<+->]
\frametitle{주요 학습내용}
    \begin{itemize}
        \item 자유의 원칙이 정치적 권위의 강제적인 행사, 재산권 및 계약 권리의 근거 및 범위와 어떻게 일치하는가.
        \item 왜 자유지상주의자들이 기회의 평등 추구나 부와 소득의 공정한 분배가 부당하고 부당하다고 생각하는 경향이 있는가.
    \end{itemize}
\end{frame}

\begin{frame}[<+->]
\frametitle{국가의 크기}
    \begin{itemize}
        \item 자유지상주의자들이 이야기 하는 작은 국가의 기준은 무엇인가?
        \begin{itemize}
            \item 국가의 규모는 지출 측면에서(예: GDP대비 정부부채).
            \item 공공 고용 측면에서(예: 총인구 대비 공무원 수).
            \item 구성원의 생활에 대한 규제의 갯수.
        \end{itemize}
        \item Nozick은 권위의 범위 측면에서 국가의 크기를 정의.
        \begin{itemize}
            \item 절대 군주는 백성의 생활 전체를 통제할 수 있는 강력한 권한을 가질 수 있지만 직위나 법을 만들지 않고도 가능(군주의 뜻 하나로도 가능).
        \end{itemize}
    \end{itemize}
\end{frame}

\begin{frame}[<+->]
\frametitle{작은 국가}
    \begin{itemize}
        \item 가장 작은 국가는 주어진 영토에서 강제력 사용을 독점하여 다른 대리인이 강제력을 사용할 수 없도록 하는 국가(긴급 상황 제외).
        \item 작은 국가는 주어진 영토에서 강제 사용에 대한 독점을 행사하고 영토에 거주하는 모든 개인의 권리가 보호되도록 보장하며 그밖에는 어떤것도 하지 않는 국가.
        \begin{itemize}
            \item 공공재, 빈민구제, 의료, 교육마저도 제공하지 않음. 단, 사회구성원의 (자유)권리를 보장하기 위해 필요한 경우는 예외.
            \item 구성원의 원래 권리와 획득한 권리를 확보하는 것 이외의 목표를 추구하는 국가는 작은 국가보다 "더 큰" 국가.
        \end{itemize}
    \end{itemize}
\end{frame}

\section{무정부주의와의 구분}

\begin{frame}[<+->]
\frametitle{무정부주의 vs. 자유지상주의}
    \begin{itemize}
        \item  무정부주의자 : 국가는 누구의 권리도 침해하지 않고 존재할 수 없으며, 구성원들의 권리는 항상 국가의 행동에 의해 침해된다.
        \begin{itemize}
            \item 정치적 권위의 기원 : 권리의 침해가 없는 국가의 정치적 권위의 불가능.
            \item 권한의 행사 : 국가 운영에는 권리 침해가 수반.
        \end{itemize}
        \item Nozick : 정치철학의 근본적인 문제는 국가를 조직하는 방법이 아니라 “국가가 존재해야 하는지의 여부.”
        \begin{itemize}
            \item Nozick은 무정부 상태에서 허용되는(즉, 권리를 침해하지 않는) 행위에 의해서만 국가가 생성될 수 있는 방법을 제시.
        \end{itemize}
    \end{itemize}
\end{frame}

\begin{frame}[<+->]
\frametitle{자연의 상태(State of nature)I}
    \begin{itemize}
        \item 자연 상태는 인간이 사회 상태에 들어가기 이전의 (시간적으로, 이론적으로, 정치적으로) 가상의 상황.
        \item 홉스는 자연 상태를 정의의 규범을 정의하고 집행할 공적 권한이 없는 상태로 설명.
        \begin{itemize}
           \item  인간의 성향, 욕구, 물질적 상황, 공권력의 부재가 결합된 상황에서 자연 상태는 필연적으로 \textbf{만인에 대한 만인의 전쟁 상태}가 되어 문화, 과학, 기술의 발전이 불가능하고, 대부분의 사람들의 삶은 “추잡하고, 고독하고, 가난하고, 야만적이며, 짧다."
        \end{itemize}
    \end{itemize}
\end{frame}

\begin{frame}[<+->]
\frametitle{자연의 상태(State of nature)II}
    \begin{itemize}
        \item 루소는 자연상태에 있는 인간의 욕망과 경향은 인간 사회의 고정된 성품이 아니라 그 자체로 인간 사회의 산물이라고 주장.
        \item 로크는 정의의 규범이 사회의 상태뿐만 아니라 자연 상태에서도 의미가 있다고 주장.
        \begin{itemize}
            \item 자연의 상태는 스스로를 돌보고 타인의 권리(생명, 자유 등)를 존중하도록 요구하는 자연 법칙의 지배를 받음.
        \end{itemize}
        \item 칸트에게 자연 상태는 다른 사람의 재산이나 서비스에 대한 권리가 생성되기 이전의 상황.
        \begin{itemize}
            \item 개인은 자신의 자연적 권리(자유의 원칙에서 직접적으로 도출되는 권리)만 가짐.
        \end{itemize}
    \end{itemize}
\end{frame}

\begin{frame}[<+->]
\frametitle{자연의 상태(State of nature)III}
    \begin{itemize}
        \item Nozick은 자연 상태에 대한 로크의 관점을 수용.
        \item  법률, 경찰, 법원 등 정치적 권위가 없는 로크의 자연 상태를 가정.
        \begin{itemize}
           \item  자연 상태에 있는 사람들은 나쁜 본성이 없음; 비정상적으로 어리석거나, 탐욕스럽거나, 악의적이거나, 허영심이 많거나, 질투하거나 시기하는 경향 등등.
           \item  특별히 좋은 본성 역시 없음; 초인간적인 지능이나 지식, 천사같은 미덕 등등.
        \end{itemize}
       \item  다른 사람들의 의견에 영향을 받지도 않음. 그러나 개인들이 보통 대부분의 경우 타인의 권리를 존중하는 경향이 있다고 가정.
        \begin{itemize}
            \item 만인의 만인에 대한 투쟁과 같은 홉스식 자연 상태와 구분.
        \end{itemize}
    \end{itemize}
\end{frame}

\begin{frame}[<+->]
\frametitle{자연상태에서의 권리}
    \begin{itemize}
        \item Nozick의 자연 상태에는 두 가지 종류의 권리가 존재.
        \item 선천적 권리는 자유의 원칙에서 직접적으로 발생하는 권리.
        \begin{itemize}
           \item  (다른 사람의 권리를 침해하지 않는 한)원하는 대로 자신을 처분할 수 있는 권리, 폭력과 기만, 강도 및 절도로부터의 자유. 
        \end{itemize}
        \item 후천적 권리는 행위를 통해 얻을 수 있는 권리 : 재산권과 계약의 권리.
        \begin{itemize}
           \item  재산권은 물건을 처분할 수 있는 권리; 물건의 원래 소유(예: 수렵) 또는 선물이나 거래를 통해 다른 사람으로부터 양도함으로써 재산을 취득.
           \item  계약의 권리는 다른 사람의 행동에 대한 권리; 물물교환에서 양도에 대한 계약 권리를 가질 수 있음.
        \end{itemize}
       \item  자연상태를 넘어서는 상태에서의 모든 권리는 자유의 원칙, 재화의 소유 또는 개인 간의 계약(동의에 의함)에서 발생.
    \end{itemize}
\end{frame}

\begin{frame}[<+->]
\frametitle{국가의 필요성}
    \begin{itemize}
       \item 자연 상태는 여전히 무정부 상태.
        \begin{itemize}
           \item 자연 상태에서는 개인들에게 선천적 권리가 있고 서로의 선천적, 후천적 권리를 존중 
           \item 그럼에도 개인들이 정치적 권위에 종속되지 않는다는 의미에서 무정부 상태.
        \end{itemize}
       \item  국가가 자연상태에서 누구의 권리도 침해하지 않고 생겨날 수 있는가.
       \item  Nozick은 주장의 출발점은 자연의 상태가 개인들에게 여전히 불편함을 준다는 점.
       \begin{itemize}
           \item 개인들은 서로의 권리를 존중하지만, 자신이 어떤 권리를 가지고 있는지 무지.
           \item 다른 사람들이 자신의 권리를 존중할 것인가 라는 불확실성의 문제에 직면. 
           \item 자신의 권리를 행사하고 침해된 경우 배상을 추구하는 데 따르는 위험과 비용.
           \item 효과적인 지출을 위한 규칙, 분쟁이 발생한 경우 판결할 수 있는 기구, 정의를 집행할 경찰도 없음.
       \end{itemize}
    \end{itemize}
\end{frame}

\begin{frame}[<+->]
\frametitle{상호보호협회(mutual-protection association)I}
    \begin{itemize}
       \item 개인들은 상호 작용의 정확한 규칙을 정의하고 분쟁을 판결하며 각자의 권리를 행사하는 데 도움이 되는 독립적인 기구를 만드는 데 동의할 수 있음.
       \item  그러한 단체를 만드는 개인들은 즉각적인 자기 방어의 경우를 제외하고 타인에게 자신의 권리를 행사할 수 있는 개인의 권리를 자연스럽게 포기.
       \item  분쟁의 판결과 권리의 집행을 위한 기구를 만들기로 한 합의는 구성원 개인의 강압적 권리를 포기하고 그러한 기구의 결정을 따르기로 하는 상호 계약을 통해 이루어짐.
    \end{itemize}
\end{frame}

\begin{frame}[<+->]
\frametitle{상호보호협회(mutual-protection association)II}
    \begin{itemize}
       \item Nozick은 그러한 협회를 "상호보호협회"라고 명명. 
       \item 구성원 전체 및 각자에게 위협이 될 수 있는 협회 외부의 사람들로부터 상호 보호를 위해 형성됨.
       \item 분쟁해결 이외에도 편의성과 비용 효율성 그리고 경쟁하는 보호 협회의 비교우위를 고려할 때 시간이 지남에 따라 하나의 보호 협회가 지역에서 지배적이 될 가능성이 큼.
       \item  그러나 지배적인 보호 협회라도 국가가 되지는 못함.
        \begin{itemize}
           \item 첫째, 다른 보호 협회가 그곳에서 활동하거나 개인이 부분적인 집행 권한을 보유할 수 있기 때문에 해당 지역에서 무력 사용에 대한 독점이 없기 때문.
           \item 둘째, 지배적 보호 협회는 해당 지역에 거주하는 모든 사람을 포함하지 않고 그 구성원만 포함.
        \end{itemize}
    \end{itemize}
\end{frame}

\begin{frame}[<+->]
\frametitle{무력의 독점}
    \begin{itemize}
        \item  상호보호협회가 국가로 발전하기 위해 지배적인 협회가 무력 사용을 독점해야 함.
        \item 무력사용에 대한 독점은 분쟁해결 수단의 생성과 적용에서 기원.
        \begin{itemize}
             \item 자유는 개인이 타인에 대한 권리를 침해, 즉 분쟁과 관련하여 모두가 신뢰할 수 있고 동시에 공정한 절차(재판)를 이용할 권리를 가짐.
             \item 재판받을 권리는 다른 협회(의 회원) 또는 무소속 개인들의 권리 침해애 대해서도 적용 되어야.
             \item 지배적인 협회는 다른 협회의 절차역시 공정하고 신뢰할 수 있는지 여부를 결정하고 자신의 구성원에 대해 결정된 절차만 사용하도록 강제하며, 지배적이기 때문에 다른 협회는 이에 반하는 동일한 수준의 권한을 행사할 수 없음.
             \item 따라서 지배적인 협회가 존재하면, 자신의 영역에서 절차의 신뢰성과 공정성 기준에 대한 견해를 강요할 권한을 행사.
        \end{itemize}
    \end{itemize}
\end{frame}

\begin{frame}[<+->]
\frametitle{독점적 영토}
    \begin{itemize}
        \item  상호보호협회가 국가로 발전하기 위해 지배적인 협회는 영토에 거주하는 모든 개인을 소속시켜야 함.
        \begin{itemize}
            \item 지배적인 협회가 강제력 사용에서, 어떤 절차가 신뢰할 수 있고 공정한지를 결정하기 위해 무력과 권력의 사용을 사실상 독점을 하게 되면, 비협회원은 협회원에 대하여 자신의 권리를 행사할 수 있는 능력이 심각하게 제한됨.
             \item 지배적인 협회는 비회원이 소속 회원에 대하여 권리를 행사하는 것을 금지함.
             \item  단, 이러한 금지는 비회원이 그로 인한 불이익에 대해 보상을 받는 경우에만 허용되며, 보상은 당연히 비회원에게 기관의 보호서비스를 무료 또는 기준가 이하의 수수료로 제공하는 형태로 해야.
        \end{itemize}
    \end{itemize}
\end{frame}

\begin{frame}[<+->]
\frametitle{국가의 탄생}
    \begin{itemize}
        \item 지배적 보호 기관은 영토 내에서 무력 사용에 대한 독점권을 가지며 그 영토에 거주하는 모든 개인을 보호.
        \item 구성원의 권리를 보호하고, 침해 시 신뢰할 수 있고 공정하다고 여기는 제도에 따라 분쟁을 심판하고 처벌.
        \item 최소한의 국가는 자유의 원칙과 그 원칙에 따라 행동하는 사람들에 의해 생성될 수 있는 다양한 자연적, 계약적, 절차적 권리에 의해 위임된 기능 외에는 어떤 기능도 수행하지 않음.
        \item 최소한의 국가는 무정부상태와 다름;  
        \begin{itemize}
            \item 국가 강압적의 행사는 개인의 권리를 계약을 통해 국가에 이전함으로써 발생한다는 점에서 개인의 자기 보호적 강압 활동과 구분.
        \end{itemize}
    \end{itemize}
\end{frame}

\section{자격 이론}
\begin{frame}[<+->]
\frametitle{소유의 정의}
    \begin{itemize}
        \item Nozick은 정부가 구성원들에게 분배해야할 대상 그 자체가 없다고 주장.
        \item 분배는 어떤 정당한 분배로부터 합법적인 수단을 통해서 이뤄질 때만 정당.
        \item 합법적인 수단은 자원의 최초 획득 그리고 재화와 용역의 이전을 규정짓는 원칙에 의해 지정.
    \end{itemize}
\end{frame}


\begin{frame}[<+->]
\frametitle{자격 이론}
    \begin{itemize}
        \item Nozick의 분배 정의는 자격의 문제; 자격을 갖추 소유만이 정의.
        \begin{itemize}
            \item 정당한 획득의 원칙 : 소유된 역사가 없는 자원의 원초적 할당에 대하여.
            \item 정당한 이전의 원칙 : 재화와 용역의 개인간 교환에 대하여.
            \item 교정의 원칙 : 위 두 원칙이 위반될 경우에 대하여.
        \end{itemize}
    \end{itemize}
\end{frame}
  
\begin{frame}[<+->]
\frametitle{정당한 획득의 원칙}
    \begin{itemize}
        \item 자연의 상태에는 다양한 (천연)자원이 존재.
        \item 인간의 능력(노동력)과 결합하여 새로운 자원으로 전환.
        \item 자원의 소유에는 두 가지 조건이 존재 :
        \begin{itemize}
            \item 개인은 자신이 훌륭하게 사용할 수 있는 정도 이상을 가질 수 없음.
            \item 타인을 위하여 충분하면서 훌륭한 잔여물이 존재해야.
        \end{itemize}
        \item 원초적 소유는 타인의 상황을 악화 시키지 말아야.
    \end{itemize}
\end{frame}

\begin{frame}[<+->]
\frametitle{정당한 이전의 원칙}
    \begin{itemize}
        \item 정당한 이전은 당사자의 자발적 선택을 알린다는 점에서 하는 자유의 원칙.
        \item 정당한 이전을 통해 물건을 얻는다면, 정당한 획득을 만족.
        \item 정당햔 이전이 반복되는 한 모든 획득은 정당함.
        \item 정당한 획득의 조건은 타인의 상황을 악화시키지 말아야 하고, 이는 정당한 이전의 원칙에 대한 제약으로 작용(예 : 물과 같은 필수재의 독점).
    \end{itemize}
\end{frame}

\begin{frame}[<+->]
\frametitle{교정의 원칙}
    \begin{itemize}
        \item 교정의 원칙은 단순하면서 강력함; 불의는 교정되어야 한다.
        \item 그렇다면 불의란 무엇인가?
        \item 모든 부당한 이전(예: 장물은 취득도 불법).
        \item 그러나 Nozick은 부당한 이전이 어디까지 거슬러 올라가야 하는가에 대해서는 말하지 않음.
        \begin{itemize}
            \item 각국의 수탈당한 해외 문화재.
            \item 식민지 경험에 대한 보상 등등.
        \end{itemize}
    \end{itemize}
\end{frame}

\begin{frame}[<+->]
\frametitle{정의로운 분배 : 자유, 소유권, 계약}
    \begin{itemize}
        \item Nozick은 소유에 대한 개인의 권리라는 관점에서 분배적 정의를 설명.
        \item 정당한 획득 또는 정당한 이전을 통한 획득만이 소유권을 인정받음.
        \item 이 과정을 통한 소유가 아닌 경우, 교정의 원칙이 작동하여 현재의 소유를 부당하게 하는 모든 불의의 교정을 강제.
        \item 교정의 원칙이 충족되고 과거에 교정이 필요한 부당함이 현재 소유로 이어지지 않으면 현재 소유는 정당하며, 모든 개인은 자신이 소유를 인정받을 자격을 획득.
    \end{itemize}
\end{frame}

\begin{frame}[<+->]
\frametitle{정의로운 분배 : 자유, 소유권, 계약}
    \begin{itemize}
        \item 하나의 분배가 다른 분배상태에서 이전에 의한 결과라면, 어떤 여건에 관계없이 그 자체로 정당.
        \item 빈곤 또는 불평등의 정도, 개인의 특성이나 운, 행복감 등등의 어떤 요소도 무관.
        \item 자원의 분배가 정당성은 소유자의 자격을 통해서만 판단.
        \item 분배의 정의는 분배의 역사의 결과이며, 그 역사가 정의롭다면 분배도 정의로움.
    \end{itemize}
\end{frame}

\section{작은 국가}

\begin{frame}[<+->]
\frametitle{재분배에 대한 시각}
    \begin{itemize}
        \item Nozick은 모든 재분배를 통한 분배적 정의의 개선시도는 부당하다고 주장.
        \begin{itemize}
            \item 개인은 정당한 소유물에 대한 권리가 존재.
            \item 개인은 소유권을 가진 물건에 대하여 사용 및 처분권 역시 존재.
            \begin{itemize}
                \item 예를들어 개인이 자신의 칼을 팔거나, 창고에 두거나, 버릴 수 있지만, 타인에게 상처 입히거나 타인의 소유물을 상하게 할 수 없음.
                \item 타인이 소유자의의 의지에 반해 칼을 가져간다면 절도이며 따라서 부당.
            \end{itemize}
        \end{itemize}
    \end{itemize}
\end{frame}

\begin{frame}[<+->]
\frametitle{작은 국가}
    \begin{itemize}
        \item 작은 국가 이상의 기능을 가진 국가를 만들려는 모든 시도는 일부 사회 구성원의 권리를 침해.
        \begin{itemize}
            \item 개인의 재산을 가져가는 사람이 타인이든 공무원이든 개인에게는 동일.
            \item 따라서 재산을 빼앗는 것과 관련된 (재)분배적 정의를 추구하는 것은 단순히 절도이며 권리에 대한 부당한 침해.
            \item 국가 기능은 과세를 통해서만 얻을 수 있는 자금이 필요하기 때문이고, 따라서 구성원의 재산권을 침해.
            \item 재분배적 과세는 "강제 노동과 동등"하다고 표현.
            \item 사회 구성원에게 일하고 세금을 내는 것에 대한 대안이 없기 때문.
        \end{itemize}
    \end{itemize}
\end{frame}

\begin{frame}[<+->]
\frametitle{분배정의에 대한 Nozick의 분류}
    \begin{itemize}
        \item 경향성의 유무.
        \begin{itemize}
            \item 키, 나이, 단순 평등, 신분 계층, 요구(needs)에 따른 분배는 유형이 있는 분배.
            \item 자유나 효용의 원칙은 경향성이 없는 원칙.
        \end{itemize}
        \item 최종상태/이력(history)
        \begin{itemize}
            \item 분배적 정의의 요소로 시간 개념을 도입.
            \item 역사적 원칙에 따르면 분배가 정당한지 아닌지는 그 역사에 달려 있음.
            \item 최종상태의 원칙은 비역사적.
            \begin{itemize}
                \item 평등주의 원칙은 분배가 정당성을 오직 현재상태로만 판단.
                \item 공리주의는 미래의 예상되는 결과를 고려하여 현재의 분배를 판단.
            \end{itemize}
        \end{itemize}
        \item  Nozick의 자격 이론은 경향성이 없으며 동시에 역사적.
    \end{itemize}
\end{frame}

\begin{frame}[<+->]
\frametitle{자유 vs. 경향성}
    \begin{itemize}
        \item 자유는 경향성을 파괴하기 때문에 둘을 모두 만족 할 수 없음.
        \begin{itemize}
            \item 평등주의 분배 vs. 자발적 거래.
            \item 평등사회에서 뛰어난 개인이 등장 한다면?
            \item 시장에서 개인의 자발적인 거래는 모든 경향성을 파괴하고 대신 욕망, 재능, 장점, 노력 및 행운의 복잡한 기능에 따라 다양한 분배를 실현시키기 때문.
        \end{itemize}
        \item 자유와 경향성 사이에 양자택일의 문제 발생.
        \begin{itemize}
            \item 우리가 자유의 원칙을 받아들인다면 분배 정의의 모든 경향성 있는 원칙을 포기해야.
            \item 효용의 원칙은 비역사적이지만 경향성은 없음.
            \item 하지만 효용의 원칙 역시 자유를 저해할 수 있음.
        \end{itemize}
    \end{itemize}
\end{frame}

\begin{frame}[<+->]
\frametitle{자유 지상주의 vs. 우파적 자유주의}
    \begin{itemize}
        \item Hayek : 자유를 누릴때 선의 총량이 극대화; 자유와 효용의 결합.
        \item Nozick : 자유가 선의 총량을 극대화 하는 특성은 없고, 오히려 양자택일의 상황이 존재.
        \begin{itemize}
            \item 탈모제 vs.에이즈 신약, 의사 vs. 연예인.
        \end{itemize}
        \item 분배 정의의 경향성 있는 원칙을 추구하는 것은 일반적 자유와 구체적인 개인의 권리를 모두 침해하며 허용되지 않음.
    \end{itemize}
\end{frame}

\begin{frame}[<+->]
\frametitle{자유지상주의와 기회}
    \begin{itemize}
        \item 평등한 기회의 원칙에도 마찬가지로 적용 됨.
        \begin{itemize}
            \item 국가가 어떤 종류의 차별을 행하는 것 역시 불의; 국가는 영토에 거주하는 모든 사람에게 동등하게 권리를 보장할 의무가 있음.
            \item 그러나, 평등한 기회를 추구하려면 더 많은 기회를 가진 사람들의 상황을 악화시키거나 기회가 적은 사람들의 상황을 개선해야.
            \item 개인에게 자격이 있는 소유는 다른 사람들에게 기회의 평등을 제공하기 위해서 몰수할 수 없음.
        \end{itemize}
    \end{itemize}
\end{frame}

\begin{frame}[<+->]
\frametitle{자유 지상주의 vs. 좌파적 자유주의I}
    \begin{itemize}
        \item 롤즈의 정의 이론은 정확하게 최종 상태만을 고려하는 이론이 아님. 
        \begin{itemize}
            \item 공정함으로서의 정의의 원칙에 의해 잘 정돈된 사회에서 분배의 정의는 역사의 결과.
            \item 개인들은 그 사회의 규칙에 따라 행동하여 재산을 얻었다면 소유에 대한 자격이 있음.
        \end{itemize}
        \item 롤즈의 정의 이론은 경향성만으로 구분할 수 없음.
        \begin{itemize}
             \item 롤즈의 정의 원칙은 기본적인 자유, 기회의 공정한 평등을 보장하고 시간이 지남에 따라 부와 소득의 분배가 가장 적게 가진 사람들에게 유리하게 되는 경향이 있는 제도의 설계.
             \item 차등의 원칙은 부와 소득의 분배가 따라야 하는 경향성을 규정하는 원칙이 아님.
             \item 구성원이 사회의 경제 제도의 설계에 대해 어떻게 생각해야 하는지를 규정.
        \end{itemize}
    \end{itemize}
\end{frame}

\begin{frame}[<+->]
\frametitle{자유 지상주의 vs. 좌파적 자유주의II}
    \begin{itemize}
        \item 롤즈의 정의 이론은 사회 구성원의 권리 침해를 정당화하지 않고,시간이 지남에 따라 공정성을 보장하기 위해 자격 제도를 설계하도록 유도.
        \begin{itemize}
            \item 정의가 공평하게 규정한 사회에서 개인들은 시간이 지남에 따라 그러한 거래의 누적 결과가 공정하다는 것을 보장하도록 설계된 제도에서 발생하는 자발적 거래에 의해 소유물을 얻을 자격을 획득.
            \item 시장에서 자발적인 거래에 참여하는 개인들의 결과인 분배는 평등이나 효용 등의 경향성을 가지더라도 공정하고 정당.
        \end{itemize}
        \item 개인들은 세후 소득과 세금을 뺀 재산 가치에 대한 자격을 얻음.
        \begin{itemize}
            \item 소득이나 재산에 대해 세금을 부과받는 경우, 이 세금은 자격에서 제외되는 것으로 간주.
            \item 세금은 개인들이 받을 자격이 있는 것을 정의하는 제도의 일부.
        \end{itemize}
    \end{itemize}
\end{frame}

\begin{frame}[<+->]
\frametitle{자유 지상주의 vs. 좌파적 자유주의(권리론)}
    \begin{itemize}
        \item 자유지상주의자들은 자격이 사회보다 우선하며 사회는 개인들이 가진 기존 자격을 보호하고 확보하기 위해 만들어졌다고 생각.
        \begin{itemize}
             \item 개인에게는 국가가 보호하고 존중해야 하는 재산권과 계약을 체결할 권리가 존재.
        \end{itemize}
        \item 좌파적 자유주의자들은 사회의 핵심이 기존의 자격이나 권리를 확보하는 것이라는 주장에 이견.
        \begin{itemize}
            \item 사회가 시간이 지남에 따라 분배의 공정성을 보장할 수 있는 자격 제도를 만들어야.
            \item 개인들이 가져야 하는 재산권 및 계약권은 공정하고 효율적인 경제를 보장하기 위해 제도를 가장 잘 설계할 수 있는 방법에 달려.
        \end{itemize}
        \item 좌파적 자유주의자에게 사회는 공정하고 효율적인 사회적 협력을 가능하게 하기 위해 존재하며, 권리와 제도는 시간이 지남에 따라 공정한 협력을 촉진하고 보장하도록 설계되어야.
    \end{itemize}
\end{frame}

\end{document}
