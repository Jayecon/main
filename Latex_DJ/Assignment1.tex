%%%%%%%%%%%%%%%%%%%%%%%%%%%%%%%%%%%%%%%%%%%%%%%%%%%%%%%%%%%%%%%
%
% Welcome to Overleaf --- just edit your LaTeX on the left,
% and we'll compile it for you on the right. If you open the
% 'Share' menu, you can invite other users to edit at the same
% time. See www.overleaf.com/learn for more info. Enjoy!
%
%%%%%%%%%%%%%%%%%%%%%%%%%%%%%%%%%%%%%%%%%%%%%%%%%%%%%%%%%%%%%%%

% ===============================================
% MATH 790: Real Analysis           Spring 2022
% hw_template.tex
% ===============================================

% -------------------------------------------------------------------------
% The preamble that follows can be ignored. Go on
% down to the section that says "START HERE" 
% -------------------------------------------------------------------------

\documentclass{article}

\usepackage[margin=1in]{geometry} 
\usepackage{amsmath,amsthm,amssymb,hyperref}
\usepackage[hangul]{kotex}
\usepackage[shortlabels]{enumitem}
\usepackage{booktabs, multicol, multirow} % Allows the use of \toprule, \midrule and \bottomrule in tables

\newcommand{\R}{\mathbf{R}}  
\newcommand{\Z}{\mathbf{Z}}
\newcommand{\N}{\mathbf{N}}
\newcommand{\Q}{\mathbf{Q}}

\renewcommand{\labelenumii}{\arabic{enumi}.\arabic{enumii}}
\renewcommand{\labelenumiii}{\arabic{enumi}.\arabic{enumii}.\arabic{enumiii}}
\renewcommand{\labelenumiv}{\arabic{enumi}.\arabic{enumii}.\arabic{enumiii}.\arabic{enumiv}}

\newenvironment{theorem}[2][Theorem]{\begin{trivlist}
\item[\hskip \labelsep {\bfseries #1}\hskip \labelsep {\bfseries #2.}]}{\end{trivlist}}
\newenvironment{lemma}[2][Lemma]{\begin{trivlist}
\item[\hskip \labelsep {\bfseries #1}\hskip \labelsep {\bfseries #2.}]}{\end{trivlist}}
\newenvironment{exercise}[2][Exercise]{\begin{trivlist}
\item[\hskip \labelsep {\bfseries #1}\hskip \labelsep {\bfseries #2.}]}{\end{trivlist}}
\newenvironment{problem}[2][Problem]{\begin{trivlist}
\item[\hskip \labelsep {\bfseries #1}\hskip \labelsep {\bfseries #2.}]}{\end{trivlist}}
\newenvironment{question}[2][Question]{\begin{trivlist}
\item[\hskip \labelsep {\bfseries #1}\hskip \labelsep {\bfseries #2.}]}{\end{trivlist}}
\newenvironment{corollary}[2][Corollary]{\begin{trivlist}
\item[\hskip \labelsep {\bfseries #1}\hskip \labelsep {\bfseries #2.}]}{\end{trivlist}}

\newenvironment{solution}{\begin{proof}[Solution]}{\end{proof}}

\begin{document}

% ------------------------------------------ %
%                 START HERE                  %
% ------------------------------------------ %

\title{과제 1} % Replace with appropriate title
\author{경제정의와 불평등} % Replace "Author's Name" with your name
\date{2022년 4월 11일 23시 59분까지 제출}

\maketitle

% -----------------------------------------------------
% The following two environments (theorem, proof) are
% where you will enter the statement and proof of your
% first problem for this assignment.
%
% In the theorem environment, you can replace the word
% "theorem" in the \begin and \end commands with
% "exercise", "problem", "lemma", etc., depending on
% what you are submitting. 
% -----------------------------------------------------
\begin{enumerate}[{\bf 문제 \arabic*.}]
    \item 분배적 정의에 대한 다음의 사상을 3문장 이내로 설명하고, 각각의 사상을 대표하는 사회후생함수를 수식과 그림으로 제시하시오.
        \begin{enumerate}
            \item 공리주의
            \item 평등주의
            \item 롤즈주의
        \end{enumerate}
    \item 표 \ref{tab:scov}\는 세 가지 대안 \{A,B,C\}에 대하여 6명의 선호를 나타낸 것이다. 
        \begin{table}[htbp]
            \centering
            \begin{tabular}{c|c|c|c|c|c|c}
                \toprule
                              & 투표자1 & 투표자2 & 투표자3 & 투표자4 & 투표자5 & 투표자6 \\
                \hline 후보 A & 6 & 4 & 5 & 7 & 0 & 4 \\
                       후보 B & 3 & 3 & 3 & 2 & 7 & 1 \\
                       후보 C & 1 & 3 & 2 & 1 & 3 & 5 \\
                \bottomrule
            \end{tabular}
            \caption{후보에 대한 투표자의 선호}
            \label{tab:scov}
        \end{table}
        \begin{enumerate}
            \item 1인 1표제 하에서 어떤 후보가 승리할 것인가?
            \item 10점의 점수투표제 하에서 모두가 진실된 선호를 이야기 할 경우의 승리자는 누구인가?
            \item 10점의 점수투표제 하에서 전략적 투표의 가능성이 있는가?
        \end{enumerate}
    \item X,Y,Z 세 사람이 점심 면요리를 고르는 문제를 생각하자. 
        \begin{table}[htbp]
            \centering
            \begin{tabular}{cc}
                 X: & 라면 $\succ$ 파스타 $\succ$ 쌀국수 \\
                 Y: & 쌀국수 $\succ$ 라면 $\succ$ 파스타 \\
                 Z: & 파스타 $\succ$ 쌀국수 $\succ$ 라면
            \end{tabular}
            \caption{개인들의 선호}
        \end{table}
        \begin{enumerate}
            \item 꽁도세 역설(Condorcet winner)을 위의 선호를 예를 들어 설명하시오.
            \item 만약 Z 사람이 라면을 쌀국수 보다 더 좋아하는 취향으로 바꼈다고 하자. 이때 꽁도세 승자(Condorcet winner)는 존재하는지 설명하시오.
        \end{enumerate}
    \item A국은 다음과 같은 부의 소득세제를 시행하고 있다. 자기가 번 소득이 100만원 이하인 사람을 대상으로 자기 소득 1만원 증가에 대하여 처분가능 소득은 7천5백원이 늘어나도록 되어 있다. 이 경우 스스로 번 소득과 처분가능 소득 사이의 관계를 식과 그림으로 나타내 보시오.
    \item 정부지출에 의한 재분배 정책은 혜택의 전가와 정책에 대한 시장의 반응 정도에 따라 상이한 효과를 가져온다. 이 가운데 혜택의 전가에 의해 의도치 않은 결과를 가져오는 예시를 5문장 이내로 제시하시오.
    \item 표 \ref{tab:incomed}는 두 사회의 구성원의 소득을 조사한 자료이다.
        \begin{table}[htbp]
            \centering
            \begin{tabular}{c|c|c|c|c|c|c}
                \toprule
                   사회 A & 1 & 2 & 3 & 8 & 20 & 50 \\
                   사회 B & 4 & 10 & 12 & 13 & 18 & 25 \\
                \bottomrule
            \end{tabular}
            \caption{소득조사}
            \label{tab:incomed}
        \end{table}
        \begin{enumerate}
            \item 각각의 사회의 로렌츠 곡선을 그리고, 로렌츠 지배 여부를 판단하라.
            \item 두 사회의 지니계수를 각각 구하라.
            \item 공리주의적 관점과 롤즈주의적 관점에서 어떤 사회가 더 나은가?
        \end{enumerate}
\end{enumerate}



% ---------------------------------------------------
% Anything after the \end{document} will be ignored by the typesetting.
% ----------------------------------------------------

\end{document}