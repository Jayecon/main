%%%%%%%%%%%%%%%%%%%%%%%%%%%%%%%%%%%%%%%%%%%%%%%%%%%%%%%%%%%%%%%
%
% Welcome to Overleaf --- just edit your LaTeX on the left,
% and we'll compile it for you on the right. If you open the
% 'Share' menu, you can invite other users to edit at the same
% time. See www.overleaf.com/learn for more info. Enjoy!
%
%%%%%%%%%%%%%%%%%%%%%%%%%%%%%%%%%%%%%%%%%%%%%%%%%%%%%%%%%%%%%%%

% ===============================================
% MATH 790: Real Analysis           Spring 2022
% hw_template.tex
% ===============================================

% -------------------------------------------------------------------------
% The preamble that follows can be ignored. Go on
% down to the section that says "START HERE" 
% -------------------------------------------------------------------------

\documentclass{article}

\usepackage[margin=1in]{geometry} 
\usepackage{amsmath,amsthm,amssymb,hyperref,graphicx}
\usepackage[hangul]{kotex}
\usepackage[shortlabels]{enumitem}
\usepackage{booktabs, multicol, multirow} % Allows the use of \toprule, \midrule and \bottomrule in tables

\newcommand{\R}{\mathbf{R}}  
\newcommand{\Z}{\mathbf{Z}}
\newcommand{\N}{\mathbf{N}}
\newcommand{\Q}{\mathbf{Q}}

\renewcommand{\labelenumii}{\arabic{enumi}.\arabic{enumii}}
\renewcommand{\labelenumiii}{\arabic{enumi}.\arabic{enumii}.\arabic{enumiii}}
\renewcommand{\labelenumiv}{\arabic{enumi}.\arabic{enumii}.\arabic{enumiii}.\arabic{enumiv}}

\newenvironment{theorem}[2][Theorem]{\begin{trivlist}
\item[\hskip \labelsep {\bfseries #1}\hskip \labelsep {\bfseries #2.}]}{\end{trivlist}}
\newenvironment{lemma}[2][Lemma]{\begin{trivlist}
\item[\hskip \labelsep {\bfseries #1}\hskip \labelsep {\bfseries #2.}]}{\end{trivlist}}
\newenvironment{exercise}[2][Exercise]{\begin{trivlist}
\item[\hskip \labelsep {\bfseries #1}\hskip \labelsep {\bfseries #2.}]}{\end{trivlist}}
\newenvironment{problem}[2][Problem]{\begin{trivlist}
\item[\hskip \labelsep {\bfseries #1}\hskip \labelsep {\bfseries #2.}]}{\end{trivlist}}
\newenvironment{question}[2][Question]{\begin{trivlist}
\item[\hskip \labelsep {\bfseries #1}\hskip \labelsep {\bfseries #2.}]}{\end{trivlist}}
\newenvironment{corollary}[2][Corollary]{\begin{trivlist}
\item[\hskip \labelsep {\bfseries #1}\hskip \labelsep {\bfseries #2.}]}{\end{trivlist}}

\newenvironment{solution}{\begin{proof}[Solution]}{\end{proof}}

\begin{document}

% ------------------------------------------ %
%                 START HERE                  %
% ------------------------------------------ %

\title{경제정의와 불평등 중간고사} % Replace with appropriate title
\author{담당교수 : 오성재} % Replace "Author's Name" with your name
\date{\today}

\maketitle

% -----------------------------------------------------
% The following two environments (theorem, proof) are
% where you will enter the statement and proof of your
% first problem for this assignment.
%
% In the theorem environment, you can replace the word
% "theorem" in the \begin and \end commands with
% "exercise", "problem", "lemma", etc., depending on
% what you are submitting. 
% -----------------------------------------------------
\begin{enumerate}[{\bf 문제 \arabic*.}]
    \item 자유지상주의의 정의관을 엄밀하게 정의 하시오.
    
    \item 한남시는 젠트리피케이션 문제로 고통받는 A지역의 상권을 활성화 하기 위하여 인근에 시 예산으로 공원을 짓는 사업을 시행하려 한다. 이때 발생할 수 있는 문제점을 수업시간에 배운 개념을 이용하여 설명하시오.
    
    \item 한남시 시의회는 복지정책의 일환으로 다음의 두 정책 가운데 한 가지에 예산을 편성하려 한다. 
        \begin{itemize}
            \item 한부모가정의 주거 안정을 위하여 전세자금을 낮은 이자율로 대출해준다.
            \item 저소득 청소년들의 건강한 성장을 위하여 방학중에도 식비를 보조하는 바우처 제도를 시행한다. 
        \end{itemize}
        수업시간에 배운 개념을 바탕으로 시의원들은 어떤 예산안을 통과시킬 지에 대하여 논하시오.

    \item 놀이공원인 L월드는 모든 이용객에게 기본 입장료를 받는다. 인기있는 놀이기구는 주요 연휴에는 최대 2시간 이상을 대기해야 이용할 수 있다. 이에 놀이공원은 추가요금을 지불하면 대기없이 기구를 이용할 수 있는 우선권을 판매하고 있다. 놀이공원이 즐거움이 아닌 시간을 판매하는 문제에 대하여 다음의 질문에 최대한 정확하게 답하시오(질문에 대한 답이 아닌 답의 근거만을 평가 하겠음).
        \begin{enumerate}
            \item 우파적 자유주의의 입장에서 이는 정의로운 분배인가?
            \item 좌파적 자유주의의 입장에서 이는 정의로운 분배인가?
            \item 자유지상주의의 입장에서 이는 정의로운 분배인가?
        \end{enumerate}
    

\pagebreak

    \item 표 \ref{tab:incomed}는 세 사회의 구성원의 소득을 조사한 자료이다.
        \begin{table}[htbp]
            \centering
            \begin{tabular}{c|c|c|c}
                \toprule
                   사회 A & 3 & 4 & 5 \\
                   사회 B & 12 & 4 & 7 \\
                   사회 C & 16 & 17 & 8 \\
                \bottomrule
            \end{tabular}
            \caption{소득조사}
            \label{tab:incomed}
        \end{table}
        \begin{enumerate}
            \item 각각의 사회의 로렌츠 곡선을 그리고, 로렌츠 지배 여부를 판단하라.
            \item 각 사회의 지니계수를 각각 구하라.
        \end{enumerate}
        
    \item 표 \ref{tab:income}는 두 사회의 구성원의 소득을 조사한 자료이다.
        \begin{table}[htbp]
            \centering
            \begin{tabular}{c|c|c|c|c}
                \toprule
                   사회 D & 3 & 4 & 5 & 7 \\
                   사회 E & 2 & 3 & 6 & 9 \\ 
                \bottomrule
            \end{tabular}
            \caption{소득조사}
            \label{tab:income}
        \end{table}
        \begin{enumerate}
            \item 공리주의, 우파적 자유주의(각 개인소득의 단순 곱), 롤즈주의 각각에서 어떤 사회가 더 나은가?
            \item 공리주의, 평등주의, 롤즈주의에서 균등분배 대등소득을 구하시오.
            \item 우파적 자유주의와 롤즈주의 각각의 입장에서 애킨슨 지수를 구하시오.
        \end{enumerate}
        
    \item 팔마비율(Palma ratio)를 정의하고 그 특성에 대하여 설명하시오.
        \begin{enumerate}
            \item 평준화 특성에 대하여 정의하고, 팔마비율이 그 특성을 만족하는지 설명하시오.
            \item 피구달튼 원칙에 대하여 정의하고, 팔마비율이 그 특성을 만족하는지 설명하시오.
        \end{enumerate}

    

\vspace{3cm}

    \centering
    \large{한 학기동안 고생하셨습니다.}
  
\end{enumerate}



% ---------------------------------------------------
% Anything after the \end{document} will be ignored by the typesetting.
% ----------------------------------------------------

\end{document}